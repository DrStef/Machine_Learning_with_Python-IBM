\documentclass[11pt]{article}

    \usepackage[breakable]{tcolorbox}
    \usepackage{parskip} % Stop auto-indenting (to mimic markdown behaviour)
    
    \usepackage{iftex}
    \ifPDFTeX
    	\usepackage[T1]{fontenc}
    	\usepackage{mathpazo}
    \else
    	\usepackage{fontspec}
    \fi

    % Basic figure setup, for now with no caption control since it's done
    % automatically by Pandoc (which extracts ![](path) syntax from Markdown).
    \usepackage{graphicx}
    % Maintain compatibility with old templates. Remove in nbconvert 6.0
    \let\Oldincludegraphics\includegraphics
    % Ensure that by default, figures have no caption (until we provide a
    % proper Figure object with a Caption API and a way to capture that
    % in the conversion process - todo).
    \usepackage{caption}
    \DeclareCaptionFormat{nocaption}{}
    \captionsetup{format=nocaption,aboveskip=0pt,belowskip=0pt}

    \usepackage{float}
    \floatplacement{figure}{H} % forces figures to be placed at the correct location
    \usepackage{xcolor} % Allow colors to be defined
    \usepackage{enumerate} % Needed for markdown enumerations to work
    \usepackage{geometry} % Used to adjust the document margins
    \usepackage{amsmath} % Equations
    \usepackage{amssymb} % Equations
    \usepackage{textcomp} % defines textquotesingle
    % Hack from http://tex.stackexchange.com/a/47451/13684:
    \AtBeginDocument{%
        \def\PYZsq{\textquotesingle}% Upright quotes in Pygmentized code
    }
    \usepackage{upquote} % Upright quotes for verbatim code
    \usepackage{eurosym} % defines \euro
    \usepackage[mathletters]{ucs} % Extended unicode (utf-8) support
    \usepackage{fancyvrb} % verbatim replacement that allows latex
    \usepackage{grffile} % extends the file name processing of package graphics 
                         % to support a larger range
    \makeatletter % fix for old versions of grffile with XeLaTeX
    \@ifpackagelater{grffile}{2019/11/01}
    {
      % Do nothing on new versions
    }
    {
      \def\Gread@@xetex#1{%
        \IfFileExists{"\Gin@base".bb}%
        {\Gread@eps{\Gin@base.bb}}%
        {\Gread@@xetex@aux#1}%
      }
    }
    \makeatother
    \usepackage[Export]{adjustbox} % Used to constrain images to a maximum size
    \adjustboxset{max size={0.9\linewidth}{0.9\paperheight}}

    % The hyperref package gives us a pdf with properly built
    % internal navigation ('pdf bookmarks' for the table of contents,
    % internal cross-reference links, web links for URLs, etc.)
    \usepackage{hyperref}
    % The default LaTeX title has an obnoxious amount of whitespace. By default,
    % titling removes some of it. It also provides customization options.
    \usepackage{titling}
    \usepackage{longtable} % longtable support required by pandoc >1.10
    \usepackage{booktabs}  % table support for pandoc > 1.12.2
    \usepackage[inline]{enumitem} % IRkernel/repr support (it uses the enumerate* environment)
    \usepackage[normalem]{ulem} % ulem is needed to support strikethroughs (\sout)
                                % normalem makes italics be italics, not underlines
    \usepackage{mathrsfs}
    

    
    % Colors for the hyperref package
    \definecolor{urlcolor}{rgb}{0,.145,.698}
    \definecolor{linkcolor}{rgb}{.71,0.21,0.01}
    \definecolor{citecolor}{rgb}{.12,.54,.11}

    % ANSI colors
    \definecolor{ansi-black}{HTML}{3E424D}
    \definecolor{ansi-black-intense}{HTML}{282C36}
    \definecolor{ansi-red}{HTML}{E75C58}
    \definecolor{ansi-red-intense}{HTML}{B22B31}
    \definecolor{ansi-green}{HTML}{00A250}
    \definecolor{ansi-green-intense}{HTML}{007427}
    \definecolor{ansi-yellow}{HTML}{DDB62B}
    \definecolor{ansi-yellow-intense}{HTML}{B27D12}
    \definecolor{ansi-blue}{HTML}{208FFB}
    \definecolor{ansi-blue-intense}{HTML}{0065CA}
    \definecolor{ansi-magenta}{HTML}{D160C4}
    \definecolor{ansi-magenta-intense}{HTML}{A03196}
    \definecolor{ansi-cyan}{HTML}{60C6C8}
    \definecolor{ansi-cyan-intense}{HTML}{258F8F}
    \definecolor{ansi-white}{HTML}{C5C1B4}
    \definecolor{ansi-white-intense}{HTML}{A1A6B2}
    \definecolor{ansi-default-inverse-fg}{HTML}{FFFFFF}
    \definecolor{ansi-default-inverse-bg}{HTML}{000000}

    % common color for the border for error outputs.
    \definecolor{outerrorbackground}{HTML}{FFDFDF}

    % commands and environments needed by pandoc snippets
    % extracted from the output of `pandoc -s`
    \providecommand{\tightlist}{%
      \setlength{\itemsep}{0pt}\setlength{\parskip}{0pt}}
    \DefineVerbatimEnvironment{Highlighting}{Verbatim}{commandchars=\\\{\}}
    % Add ',fontsize=\small' for more characters per line
    \newenvironment{Shaded}{}{}
    \newcommand{\KeywordTok}[1]{\textcolor[rgb]{0.00,0.44,0.13}{\textbf{{#1}}}}
    \newcommand{\DataTypeTok}[1]{\textcolor[rgb]{0.56,0.13,0.00}{{#1}}}
    \newcommand{\DecValTok}[1]{\textcolor[rgb]{0.25,0.63,0.44}{{#1}}}
    \newcommand{\BaseNTok}[1]{\textcolor[rgb]{0.25,0.63,0.44}{{#1}}}
    \newcommand{\FloatTok}[1]{\textcolor[rgb]{0.25,0.63,0.44}{{#1}}}
    \newcommand{\CharTok}[1]{\textcolor[rgb]{0.25,0.44,0.63}{{#1}}}
    \newcommand{\StringTok}[1]{\textcolor[rgb]{0.25,0.44,0.63}{{#1}}}
    \newcommand{\CommentTok}[1]{\textcolor[rgb]{0.38,0.63,0.69}{\textit{{#1}}}}
    \newcommand{\OtherTok}[1]{\textcolor[rgb]{0.00,0.44,0.13}{{#1}}}
    \newcommand{\AlertTok}[1]{\textcolor[rgb]{1.00,0.00,0.00}{\textbf{{#1}}}}
    \newcommand{\FunctionTok}[1]{\textcolor[rgb]{0.02,0.16,0.49}{{#1}}}
    \newcommand{\RegionMarkerTok}[1]{{#1}}
    \newcommand{\ErrorTok}[1]{\textcolor[rgb]{1.00,0.00,0.00}{\textbf{{#1}}}}
    \newcommand{\NormalTok}[1]{{#1}}
    
    % Additional commands for more recent versions of Pandoc
    \newcommand{\ConstantTok}[1]{\textcolor[rgb]{0.53,0.00,0.00}{{#1}}}
    \newcommand{\SpecialCharTok}[1]{\textcolor[rgb]{0.25,0.44,0.63}{{#1}}}
    \newcommand{\VerbatimStringTok}[1]{\textcolor[rgb]{0.25,0.44,0.63}{{#1}}}
    \newcommand{\SpecialStringTok}[1]{\textcolor[rgb]{0.73,0.40,0.53}{{#1}}}
    \newcommand{\ImportTok}[1]{{#1}}
    \newcommand{\DocumentationTok}[1]{\textcolor[rgb]{0.73,0.13,0.13}{\textit{{#1}}}}
    \newcommand{\AnnotationTok}[1]{\textcolor[rgb]{0.38,0.63,0.69}{\textbf{\textit{{#1}}}}}
    \newcommand{\CommentVarTok}[1]{\textcolor[rgb]{0.38,0.63,0.69}{\textbf{\textit{{#1}}}}}
    \newcommand{\VariableTok}[1]{\textcolor[rgb]{0.10,0.09,0.49}{{#1}}}
    \newcommand{\ControlFlowTok}[1]{\textcolor[rgb]{0.00,0.44,0.13}{\textbf{{#1}}}}
    \newcommand{\OperatorTok}[1]{\textcolor[rgb]{0.40,0.40,0.40}{{#1}}}
    \newcommand{\BuiltInTok}[1]{{#1}}
    \newcommand{\ExtensionTok}[1]{{#1}}
    \newcommand{\PreprocessorTok}[1]{\textcolor[rgb]{0.74,0.48,0.00}{{#1}}}
    \newcommand{\AttributeTok}[1]{\textcolor[rgb]{0.49,0.56,0.16}{{#1}}}
    \newcommand{\InformationTok}[1]{\textcolor[rgb]{0.38,0.63,0.69}{\textbf{\textit{{#1}}}}}
    \newcommand{\WarningTok}[1]{\textcolor[rgb]{0.38,0.63,0.69}{\textbf{\textit{{#1}}}}}
    
    
    % Define a nice break command that doesn't care if a line doesn't already
    % exist.
    \def\br{\hspace*{\fill} \\* }
    % Math Jax compatibility definitions
    \def\gt{>}
    \def\lt{<}
    \let\Oldtex\TeX
    \let\Oldlatex\LaTeX
    \renewcommand{\TeX}{\textrm{\Oldtex}}
    \renewcommand{\LaTeX}{\textrm{\Oldlatex}}
    % Document parameters
    % Document title
    \title{ML0101EN-Density-Based Clustering-Weather-V2}
    
    
    
    
    
% Pygments definitions
\makeatletter
\def\PY@reset{\let\PY@it=\relax \let\PY@bf=\relax%
    \let\PY@ul=\relax \let\PY@tc=\relax%
    \let\PY@bc=\relax \let\PY@ff=\relax}
\def\PY@tok#1{\csname PY@tok@#1\endcsname}
\def\PY@toks#1+{\ifx\relax#1\empty\else%
    \PY@tok{#1}\expandafter\PY@toks\fi}
\def\PY@do#1{\PY@bc{\PY@tc{\PY@ul{%
    \PY@it{\PY@bf{\PY@ff{#1}}}}}}}
\def\PY#1#2{\PY@reset\PY@toks#1+\relax+\PY@do{#2}}

\@namedef{PY@tok@w}{\def\PY@tc##1{\textcolor[rgb]{0.73,0.73,0.73}{##1}}}
\@namedef{PY@tok@c}{\let\PY@it=\textit\def\PY@tc##1{\textcolor[rgb]{0.25,0.50,0.50}{##1}}}
\@namedef{PY@tok@cp}{\def\PY@tc##1{\textcolor[rgb]{0.74,0.48,0.00}{##1}}}
\@namedef{PY@tok@k}{\let\PY@bf=\textbf\def\PY@tc##1{\textcolor[rgb]{0.00,0.50,0.00}{##1}}}
\@namedef{PY@tok@kp}{\def\PY@tc##1{\textcolor[rgb]{0.00,0.50,0.00}{##1}}}
\@namedef{PY@tok@kt}{\def\PY@tc##1{\textcolor[rgb]{0.69,0.00,0.25}{##1}}}
\@namedef{PY@tok@o}{\def\PY@tc##1{\textcolor[rgb]{0.40,0.40,0.40}{##1}}}
\@namedef{PY@tok@ow}{\let\PY@bf=\textbf\def\PY@tc##1{\textcolor[rgb]{0.67,0.13,1.00}{##1}}}
\@namedef{PY@tok@nb}{\def\PY@tc##1{\textcolor[rgb]{0.00,0.50,0.00}{##1}}}
\@namedef{PY@tok@nf}{\def\PY@tc##1{\textcolor[rgb]{0.00,0.00,1.00}{##1}}}
\@namedef{PY@tok@nc}{\let\PY@bf=\textbf\def\PY@tc##1{\textcolor[rgb]{0.00,0.00,1.00}{##1}}}
\@namedef{PY@tok@nn}{\let\PY@bf=\textbf\def\PY@tc##1{\textcolor[rgb]{0.00,0.00,1.00}{##1}}}
\@namedef{PY@tok@ne}{\let\PY@bf=\textbf\def\PY@tc##1{\textcolor[rgb]{0.82,0.25,0.23}{##1}}}
\@namedef{PY@tok@nv}{\def\PY@tc##1{\textcolor[rgb]{0.10,0.09,0.49}{##1}}}
\@namedef{PY@tok@no}{\def\PY@tc##1{\textcolor[rgb]{0.53,0.00,0.00}{##1}}}
\@namedef{PY@tok@nl}{\def\PY@tc##1{\textcolor[rgb]{0.63,0.63,0.00}{##1}}}
\@namedef{PY@tok@ni}{\let\PY@bf=\textbf\def\PY@tc##1{\textcolor[rgb]{0.60,0.60,0.60}{##1}}}
\@namedef{PY@tok@na}{\def\PY@tc##1{\textcolor[rgb]{0.49,0.56,0.16}{##1}}}
\@namedef{PY@tok@nt}{\let\PY@bf=\textbf\def\PY@tc##1{\textcolor[rgb]{0.00,0.50,0.00}{##1}}}
\@namedef{PY@tok@nd}{\def\PY@tc##1{\textcolor[rgb]{0.67,0.13,1.00}{##1}}}
\@namedef{PY@tok@s}{\def\PY@tc##1{\textcolor[rgb]{0.73,0.13,0.13}{##1}}}
\@namedef{PY@tok@sd}{\let\PY@it=\textit\def\PY@tc##1{\textcolor[rgb]{0.73,0.13,0.13}{##1}}}
\@namedef{PY@tok@si}{\let\PY@bf=\textbf\def\PY@tc##1{\textcolor[rgb]{0.73,0.40,0.53}{##1}}}
\@namedef{PY@tok@se}{\let\PY@bf=\textbf\def\PY@tc##1{\textcolor[rgb]{0.73,0.40,0.13}{##1}}}
\@namedef{PY@tok@sr}{\def\PY@tc##1{\textcolor[rgb]{0.73,0.40,0.53}{##1}}}
\@namedef{PY@tok@ss}{\def\PY@tc##1{\textcolor[rgb]{0.10,0.09,0.49}{##1}}}
\@namedef{PY@tok@sx}{\def\PY@tc##1{\textcolor[rgb]{0.00,0.50,0.00}{##1}}}
\@namedef{PY@tok@m}{\def\PY@tc##1{\textcolor[rgb]{0.40,0.40,0.40}{##1}}}
\@namedef{PY@tok@gh}{\let\PY@bf=\textbf\def\PY@tc##1{\textcolor[rgb]{0.00,0.00,0.50}{##1}}}
\@namedef{PY@tok@gu}{\let\PY@bf=\textbf\def\PY@tc##1{\textcolor[rgb]{0.50,0.00,0.50}{##1}}}
\@namedef{PY@tok@gd}{\def\PY@tc##1{\textcolor[rgb]{0.63,0.00,0.00}{##1}}}
\@namedef{PY@tok@gi}{\def\PY@tc##1{\textcolor[rgb]{0.00,0.63,0.00}{##1}}}
\@namedef{PY@tok@gr}{\def\PY@tc##1{\textcolor[rgb]{1.00,0.00,0.00}{##1}}}
\@namedef{PY@tok@ge}{\let\PY@it=\textit}
\@namedef{PY@tok@gs}{\let\PY@bf=\textbf}
\@namedef{PY@tok@gp}{\let\PY@bf=\textbf\def\PY@tc##1{\textcolor[rgb]{0.00,0.00,0.50}{##1}}}
\@namedef{PY@tok@go}{\def\PY@tc##1{\textcolor[rgb]{0.53,0.53,0.53}{##1}}}
\@namedef{PY@tok@gt}{\def\PY@tc##1{\textcolor[rgb]{0.00,0.27,0.87}{##1}}}
\@namedef{PY@tok@err}{\def\PY@bc##1{{\setlength{\fboxsep}{\string -\fboxrule}\fcolorbox[rgb]{1.00,0.00,0.00}{1,1,1}{\strut ##1}}}}
\@namedef{PY@tok@kc}{\let\PY@bf=\textbf\def\PY@tc##1{\textcolor[rgb]{0.00,0.50,0.00}{##1}}}
\@namedef{PY@tok@kd}{\let\PY@bf=\textbf\def\PY@tc##1{\textcolor[rgb]{0.00,0.50,0.00}{##1}}}
\@namedef{PY@tok@kn}{\let\PY@bf=\textbf\def\PY@tc##1{\textcolor[rgb]{0.00,0.50,0.00}{##1}}}
\@namedef{PY@tok@kr}{\let\PY@bf=\textbf\def\PY@tc##1{\textcolor[rgb]{0.00,0.50,0.00}{##1}}}
\@namedef{PY@tok@bp}{\def\PY@tc##1{\textcolor[rgb]{0.00,0.50,0.00}{##1}}}
\@namedef{PY@tok@fm}{\def\PY@tc##1{\textcolor[rgb]{0.00,0.00,1.00}{##1}}}
\@namedef{PY@tok@vc}{\def\PY@tc##1{\textcolor[rgb]{0.10,0.09,0.49}{##1}}}
\@namedef{PY@tok@vg}{\def\PY@tc##1{\textcolor[rgb]{0.10,0.09,0.49}{##1}}}
\@namedef{PY@tok@vi}{\def\PY@tc##1{\textcolor[rgb]{0.10,0.09,0.49}{##1}}}
\@namedef{PY@tok@vm}{\def\PY@tc##1{\textcolor[rgb]{0.10,0.09,0.49}{##1}}}
\@namedef{PY@tok@sa}{\def\PY@tc##1{\textcolor[rgb]{0.73,0.13,0.13}{##1}}}
\@namedef{PY@tok@sb}{\def\PY@tc##1{\textcolor[rgb]{0.73,0.13,0.13}{##1}}}
\@namedef{PY@tok@sc}{\def\PY@tc##1{\textcolor[rgb]{0.73,0.13,0.13}{##1}}}
\@namedef{PY@tok@dl}{\def\PY@tc##1{\textcolor[rgb]{0.73,0.13,0.13}{##1}}}
\@namedef{PY@tok@s2}{\def\PY@tc##1{\textcolor[rgb]{0.73,0.13,0.13}{##1}}}
\@namedef{PY@tok@sh}{\def\PY@tc##1{\textcolor[rgb]{0.73,0.13,0.13}{##1}}}
\@namedef{PY@tok@s1}{\def\PY@tc##1{\textcolor[rgb]{0.73,0.13,0.13}{##1}}}
\@namedef{PY@tok@mb}{\def\PY@tc##1{\textcolor[rgb]{0.40,0.40,0.40}{##1}}}
\@namedef{PY@tok@mf}{\def\PY@tc##1{\textcolor[rgb]{0.40,0.40,0.40}{##1}}}
\@namedef{PY@tok@mh}{\def\PY@tc##1{\textcolor[rgb]{0.40,0.40,0.40}{##1}}}
\@namedef{PY@tok@mi}{\def\PY@tc##1{\textcolor[rgb]{0.40,0.40,0.40}{##1}}}
\@namedef{PY@tok@il}{\def\PY@tc##1{\textcolor[rgb]{0.40,0.40,0.40}{##1}}}
\@namedef{PY@tok@mo}{\def\PY@tc##1{\textcolor[rgb]{0.40,0.40,0.40}{##1}}}
\@namedef{PY@tok@ch}{\let\PY@it=\textit\def\PY@tc##1{\textcolor[rgb]{0.25,0.50,0.50}{##1}}}
\@namedef{PY@tok@cm}{\let\PY@it=\textit\def\PY@tc##1{\textcolor[rgb]{0.25,0.50,0.50}{##1}}}
\@namedef{PY@tok@cpf}{\let\PY@it=\textit\def\PY@tc##1{\textcolor[rgb]{0.25,0.50,0.50}{##1}}}
\@namedef{PY@tok@c1}{\let\PY@it=\textit\def\PY@tc##1{\textcolor[rgb]{0.25,0.50,0.50}{##1}}}
\@namedef{PY@tok@cs}{\let\PY@it=\textit\def\PY@tc##1{\textcolor[rgb]{0.25,0.50,0.50}{##1}}}

\def\PYZbs{\char`\\}
\def\PYZus{\char`\_}
\def\PYZob{\char`\{}
\def\PYZcb{\char`\}}
\def\PYZca{\char`\^}
\def\PYZam{\char`\&}
\def\PYZlt{\char`\<}
\def\PYZgt{\char`\>}
\def\PYZsh{\char`\#}
\def\PYZpc{\char`\%}
\def\PYZdl{\char`\$}
\def\PYZhy{\char`\-}
\def\PYZsq{\char`\'}
\def\PYZdq{\char`\"}
\def\PYZti{\char`\~}
% for compatibility with earlier versions
\def\PYZat{@}
\def\PYZlb{[}
\def\PYZrb{]}
\makeatother


    % For linebreaks inside Verbatim environment from package fancyvrb. 
    \makeatletter
        \newbox\Wrappedcontinuationbox 
        \newbox\Wrappedvisiblespacebox 
        \newcommand*\Wrappedvisiblespace {\textcolor{red}{\textvisiblespace}} 
        \newcommand*\Wrappedcontinuationsymbol {\textcolor{red}{\llap{\tiny$\m@th\hookrightarrow$}}} 
        \newcommand*\Wrappedcontinuationindent {3ex } 
        \newcommand*\Wrappedafterbreak {\kern\Wrappedcontinuationindent\copy\Wrappedcontinuationbox} 
        % Take advantage of the already applied Pygments mark-up to insert 
        % potential linebreaks for TeX processing. 
        %        {, <, #, %, $, ' and ": go to next line. 
        %        _, }, ^, &, >, - and ~: stay at end of broken line. 
        % Use of \textquotesingle for straight quote. 
        \newcommand*\Wrappedbreaksatspecials {% 
            \def\PYGZus{\discretionary{\char`\_}{\Wrappedafterbreak}{\char`\_}}% 
            \def\PYGZob{\discretionary{}{\Wrappedafterbreak\char`\{}{\char`\{}}% 
            \def\PYGZcb{\discretionary{\char`\}}{\Wrappedafterbreak}{\char`\}}}% 
            \def\PYGZca{\discretionary{\char`\^}{\Wrappedafterbreak}{\char`\^}}% 
            \def\PYGZam{\discretionary{\char`\&}{\Wrappedafterbreak}{\char`\&}}% 
            \def\PYGZlt{\discretionary{}{\Wrappedafterbreak\char`\<}{\char`\<}}% 
            \def\PYGZgt{\discretionary{\char`\>}{\Wrappedafterbreak}{\char`\>}}% 
            \def\PYGZsh{\discretionary{}{\Wrappedafterbreak\char`\#}{\char`\#}}% 
            \def\PYGZpc{\discretionary{}{\Wrappedafterbreak\char`\%}{\char`\%}}% 
            \def\PYGZdl{\discretionary{}{\Wrappedafterbreak\char`\$}{\char`\$}}% 
            \def\PYGZhy{\discretionary{\char`\-}{\Wrappedafterbreak}{\char`\-}}% 
            \def\PYGZsq{\discretionary{}{\Wrappedafterbreak\textquotesingle}{\textquotesingle}}% 
            \def\PYGZdq{\discretionary{}{\Wrappedafterbreak\char`\"}{\char`\"}}% 
            \def\PYGZti{\discretionary{\char`\~}{\Wrappedafterbreak}{\char`\~}}% 
        } 
        % Some characters . , ; ? ! / are not pygmentized. 
        % This macro makes them "active" and they will insert potential linebreaks 
        \newcommand*\Wrappedbreaksatpunct {% 
            \lccode`\~`\.\lowercase{\def~}{\discretionary{\hbox{\char`\.}}{\Wrappedafterbreak}{\hbox{\char`\.}}}% 
            \lccode`\~`\,\lowercase{\def~}{\discretionary{\hbox{\char`\,}}{\Wrappedafterbreak}{\hbox{\char`\,}}}% 
            \lccode`\~`\;\lowercase{\def~}{\discretionary{\hbox{\char`\;}}{\Wrappedafterbreak}{\hbox{\char`\;}}}% 
            \lccode`\~`\:\lowercase{\def~}{\discretionary{\hbox{\char`\:}}{\Wrappedafterbreak}{\hbox{\char`\:}}}% 
            \lccode`\~`\?\lowercase{\def~}{\discretionary{\hbox{\char`\?}}{\Wrappedafterbreak}{\hbox{\char`\?}}}% 
            \lccode`\~`\!\lowercase{\def~}{\discretionary{\hbox{\char`\!}}{\Wrappedafterbreak}{\hbox{\char`\!}}}% 
            \lccode`\~`\/\lowercase{\def~}{\discretionary{\hbox{\char`\/}}{\Wrappedafterbreak}{\hbox{\char`\/}}}% 
            \catcode`\.\active
            \catcode`\,\active 
            \catcode`\;\active
            \catcode`\:\active
            \catcode`\?\active
            \catcode`\!\active
            \catcode`\/\active 
            \lccode`\~`\~ 	
        }
    \makeatother

    \let\OriginalVerbatim=\Verbatim
    \makeatletter
    \renewcommand{\Verbatim}[1][1]{%
        %\parskip\z@skip
        \sbox\Wrappedcontinuationbox {\Wrappedcontinuationsymbol}%
        \sbox\Wrappedvisiblespacebox {\FV@SetupFont\Wrappedvisiblespace}%
        \def\FancyVerbFormatLine ##1{\hsize\linewidth
            \vtop{\raggedright\hyphenpenalty\z@\exhyphenpenalty\z@
                \doublehyphendemerits\z@\finalhyphendemerits\z@
                \strut ##1\strut}%
        }%
        % If the linebreak is at a space, the latter will be displayed as visible
        % space at end of first line, and a continuation symbol starts next line.
        % Stretch/shrink are however usually zero for typewriter font.
        \def\FV@Space {%
            \nobreak\hskip\z@ plus\fontdimen3\font minus\fontdimen4\font
            \discretionary{\copy\Wrappedvisiblespacebox}{\Wrappedafterbreak}
            {\kern\fontdimen2\font}%
        }%
        
        % Allow breaks at special characters using \PYG... macros.
        \Wrappedbreaksatspecials
        % Breaks at punctuation characters . , ; ? ! and / need catcode=\active 	
        \OriginalVerbatim[#1,codes*=\Wrappedbreaksatpunct]%
    }
    \makeatother

    % Exact colors from NB
    \definecolor{incolor}{HTML}{303F9F}
    \definecolor{outcolor}{HTML}{D84315}
    \definecolor{cellborder}{HTML}{CFCFCF}
    \definecolor{cellbackground}{HTML}{F7F7F7}
    
    % prompt
    \makeatletter
    \newcommand{\boxspacing}{\kern\kvtcb@left@rule\kern\kvtcb@boxsep}
    \makeatother
    \newcommand{\prompt}[4]{
        {\ttfamily\llap{{\color{#2}[#3]:\hspace{3pt}#4}}\vspace{-\baselineskip}}
    }
    

    
    % Prevent overflowing lines due to hard-to-break entities
    \sloppy 
    % Setup hyperref package
    \hypersetup{
      breaklinks=true,  % so long urls are correctly broken across lines
      colorlinks=true,
      urlcolor=urlcolor,
      linkcolor=linkcolor,
      citecolor=citecolor,
      }
    % Slightly bigger margins than the latex defaults
    
    \geometry{verbose,tmargin=1in,bmargin=1in,lmargin=1in,rmargin=1in}
    
    

\begin{document}
    
    \maketitle
    
    

    
    \hypertarget{density-based-clustering}{%
\section{Density-Based Clustering}\label{density-based-clustering}}

Course: Machine Learning with Python.¶

Stephane Dedieu April May 2022 - Rev.~Oct 2022.

Estimated time needed: \textbf{25} minutes

\hypertarget{objectives}{%
\subsection{Objectives}\label{objectives}}

After completing this lab you will be able to:

\begin{itemize}
\tightlist
\item
  Use DBSCAN to do Density based clustering
\item
  Use Matplotlib to plot clusters
\end{itemize}

    Most of the traditional clustering techniques, such as k-means,
hierarchical and fuzzy clustering, can be used to group data without
supervision.

However, when applied to tasks with arbitrary shape clusters, or
clusters within a cluster, the traditional techniques might be unable to
achieve good results. That is, elements in the same cluster might not
share enough similarity or the performance may be poor. Additionally,
Density-based clustering locates regions of high density that are
separated from one another by regions of low density. Density, in this
context, is defined as the number of points within a specified radius.

In this section, the main focus will be manipulating the data and
properties of DBSCAN and observing the resulting clustering.

    Import the following libraries:

numpy as np

DBSCAN from sklearn.cluster

make\_blobs from sklearn.datasets.samples\_generator

StandardScaler from sklearn.preprocessing

matplotlib.pyplot as plt

Remember \%matplotlib inline to display plots

    \begin{tcolorbox}[breakable, size=fbox, boxrule=1pt, pad at break*=1mm,colback=cellbackground, colframe=cellborder]
\prompt{In}{incolor}{1}{\boxspacing}
\begin{Verbatim}[commandchars=\\\{\}]
\PY{c+c1}{\PYZsh{} Notice: For visualization of map, you need basemap package.}
\PY{c+c1}{\PYZsh{} if you dont have basemap install on your machine, you can use the following line to install it}
\PY{c+c1}{\PYZsh{}!pip install basemap==1.2.0 matplotlib==3.1}
\PY{c+c1}{\PYZsh{} Notice: you maight have to refresh your page and re\PYZhy{}run the notebook after installation}
\end{Verbatim}
\end{tcolorbox}

    \begin{tcolorbox}[breakable, size=fbox, boxrule=1pt, pad at break*=1mm,colback=cellbackground, colframe=cellborder]
\prompt{In}{incolor}{2}{\boxspacing}
\begin{Verbatim}[commandchars=\\\{\}]
\PY{k+kn}{import} \PY{n+nn}{numpy} \PY{k}{as} \PY{n+nn}{np} 
\PY{k+kn}{from} \PY{n+nn}{sklearn}\PY{n+nn}{.}\PY{n+nn}{cluster} \PY{k+kn}{import} \PY{n}{DBSCAN} 
\PY{k+kn}{from} \PY{n+nn}{sklearn}\PY{n+nn}{.}\PY{n+nn}{datasets} \PY{k+kn}{import} \PY{n}{make\PYZus{}blobs} 
\PY{k+kn}{from} \PY{n+nn}{sklearn}\PY{n+nn}{.}\PY{n+nn}{preprocessing} \PY{k+kn}{import} \PY{n}{StandardScaler} 
\PY{k+kn}{import} \PY{n+nn}{matplotlib}\PY{n+nn}{.}\PY{n+nn}{pyplot} \PY{k}{as} \PY{n+nn}{plt} 
\PY{o}{\PYZpc{}}\PY{k}{matplotlib} inline
\PY{k+kn}{import} \PY{n+nn}{warnings}
\PY{n}{warnings}\PY{o}{.}\PY{n}{filterwarnings}\PY{p}{(}\PY{l+s+s2}{\PYZdq{}}\PY{l+s+s2}{ignore}\PY{l+s+s2}{\PYZdq{}}\PY{p}{,} \PY{n}{category}\PY{o}{=}\PY{n+ne}{DeprecationWarning}\PY{p}{)}
\end{Verbatim}
\end{tcolorbox}

    \hypertarget{data-generation}{%
\subsubsection{Data generation}\label{data-generation}}

The function below will generate the data points and requires these
inputs:

centroidLocation: Coordinates of the centroids that will generate the
random data.

Example: input: {[}{[}4,3{]}, {[}2,-1{]}, {[}-1,4{]}{]}

numSamples: The number of data points we want generated, split over the
number of centroids (\# of centroids defined in centroidLocation)

Example: 1500

clusterDeviation: The standard deviation of the clusters. The larger the
number, the further the spacing of the data points within the clusters.

Example: 0.5

    \begin{tcolorbox}[breakable, size=fbox, boxrule=1pt, pad at break*=1mm,colback=cellbackground, colframe=cellborder]
\prompt{In}{incolor}{3}{\boxspacing}
\begin{Verbatim}[commandchars=\\\{\}]
\PY{k}{def} \PY{n+nf}{createDataPoints}\PY{p}{(}\PY{n}{centroidLocation}\PY{p}{,} \PY{n}{numSamples}\PY{p}{,} \PY{n}{clusterDeviation}\PY{p}{)}\PY{p}{:}
    \PY{c+c1}{\PYZsh{} Create random data and store in feature matrix X and response vector y.}
    \PY{n}{X}\PY{p}{,} \PY{n}{y} \PY{o}{=} \PY{n}{make\PYZus{}blobs}\PY{p}{(}\PY{n}{n\PYZus{}samples}\PY{o}{=}\PY{n}{numSamples}\PY{p}{,} \PY{n}{centers}\PY{o}{=}\PY{n}{centroidLocation}\PY{p}{,} 
                                \PY{n}{cluster\PYZus{}std}\PY{o}{=}\PY{n}{clusterDeviation}\PY{p}{)}
    
    \PY{c+c1}{\PYZsh{} Standardize features by removing the mean and scaling to unit variance}
    \PY{n}{X} \PY{o}{=} \PY{n}{StandardScaler}\PY{p}{(}\PY{p}{)}\PY{o}{.}\PY{n}{fit\PYZus{}transform}\PY{p}{(}\PY{n}{X}\PY{p}{)}
    \PY{k}{return} \PY{n}{X}\PY{p}{,} \PY{n}{y}
\end{Verbatim}
\end{tcolorbox}

    Use createDataPoints with the 3 inputs and store the output into
variables X and y.

    \begin{tcolorbox}[breakable, size=fbox, boxrule=1pt, pad at break*=1mm,colback=cellbackground, colframe=cellborder]
\prompt{In}{incolor}{4}{\boxspacing}
\begin{Verbatim}[commandchars=\\\{\}]
\PY{n}{X}\PY{p}{,} \PY{n}{y} \PY{o}{=} \PY{n}{createDataPoints}\PY{p}{(}\PY{p}{[}\PY{p}{[}\PY{l+m+mi}{4}\PY{p}{,}\PY{l+m+mi}{3}\PY{p}{]}\PY{p}{,} \PY{p}{[}\PY{l+m+mi}{2}\PY{p}{,}\PY{o}{\PYZhy{}}\PY{l+m+mi}{1}\PY{p}{]}\PY{p}{,} \PY{p}{[}\PY{o}{\PYZhy{}}\PY{l+m+mi}{1}\PY{p}{,}\PY{l+m+mi}{4}\PY{p}{]}\PY{p}{]} \PY{p}{,} \PY{l+m+mi}{1500}\PY{p}{,} \PY{l+m+mf}{0.5}\PY{p}{)}
\end{Verbatim}
\end{tcolorbox}

    \hypertarget{modeling}{%
\subsubsection{Modeling}\label{modeling}}

DBSCAN stands for Density-Based Spatial Clustering of Applications with
Noise. This technique is one of the most common clustering algorithms
which works based on density of object. The whole idea is that if a
particular point belongs to a cluster, it should be near to lots of
other points in that cluster.

It works based on two parameters: Epsilon and Minimum Points\\
\textbf{Epsilon} determine a specified radius that if includes enough
number of points within, we call it dense area\\
\textbf{minimumSamples} determine the minimum number of data points we
want in a neighborhood to define a cluster.

    \begin{tcolorbox}[breakable, size=fbox, boxrule=1pt, pad at break*=1mm,colback=cellbackground, colframe=cellborder]
\prompt{In}{incolor}{5}{\boxspacing}
\begin{Verbatim}[commandchars=\\\{\}]
\PY{n}{epsilon} \PY{o}{=} \PY{l+m+mf}{0.3}
\PY{n}{minimumSamples} \PY{o}{=} \PY{l+m+mi}{7}
\PY{n}{db} \PY{o}{=} \PY{n}{DBSCAN}\PY{p}{(}\PY{n}{eps}\PY{o}{=}\PY{n}{epsilon}\PY{p}{,} \PY{n}{min\PYZus{}samples}\PY{o}{=}\PY{n}{minimumSamples}\PY{p}{)}\PY{o}{.}\PY{n}{fit}\PY{p}{(}\PY{n}{X}\PY{p}{)}
\PY{n}{labels} \PY{o}{=} \PY{n}{db}\PY{o}{.}\PY{n}{labels\PYZus{}}
\PY{n}{labels}
\end{Verbatim}
\end{tcolorbox}

            \begin{tcolorbox}[breakable, size=fbox, boxrule=.5pt, pad at break*=1mm, opacityfill=0]
\prompt{Out}{outcolor}{5}{\boxspacing}
\begin{Verbatim}[commandchars=\\\{\}]
array([0, 1, 1, {\ldots}, 0, 1, 1], dtype=int64)
\end{Verbatim}
\end{tcolorbox}
        
    \hypertarget{distinguish-outliers}{%
\subsubsection{Distinguish outliers}\label{distinguish-outliers}}

Let's Replace all elements with `True' in core\_samples\_mask that are
in the cluster, `False' if the points are outliers.

    \begin{tcolorbox}[breakable, size=fbox, boxrule=1pt, pad at break*=1mm,colback=cellbackground, colframe=cellborder]
\prompt{In}{incolor}{6}{\boxspacing}
\begin{Verbatim}[commandchars=\\\{\}]
\PY{c+c1}{\PYZsh{} Firts, create an array of booleans using the labels from db.}
\PY{n}{core\PYZus{}samples\PYZus{}mask} \PY{o}{=} \PY{n}{np}\PY{o}{.}\PY{n}{zeros\PYZus{}like}\PY{p}{(}\PY{n}{db}\PY{o}{.}\PY{n}{labels\PYZus{}}\PY{p}{,} \PY{n}{dtype}\PY{o}{=}\PY{n+nb}{bool}\PY{p}{)}
\PY{n}{core\PYZus{}samples\PYZus{}mask}\PY{p}{[}\PY{n}{db}\PY{o}{.}\PY{n}{core\PYZus{}sample\PYZus{}indices\PYZus{}}\PY{p}{]} \PY{o}{=} \PY{k+kc}{True}
\PY{n}{core\PYZus{}samples\PYZus{}mask}
\end{Verbatim}
\end{tcolorbox}

            \begin{tcolorbox}[breakable, size=fbox, boxrule=.5pt, pad at break*=1mm, opacityfill=0]
\prompt{Out}{outcolor}{6}{\boxspacing}
\begin{Verbatim}[commandchars=\\\{\}]
array([ True,  True,  True, {\ldots},  True,  True,  True])
\end{Verbatim}
\end{tcolorbox}
        
    \begin{tcolorbox}[breakable, size=fbox, boxrule=1pt, pad at break*=1mm,colback=cellbackground, colframe=cellborder]
\prompt{In}{incolor}{7}{\boxspacing}
\begin{Verbatim}[commandchars=\\\{\}]
\PY{c+c1}{\PYZsh{} Number of clusters in labels, ignoring noise if present.}
\PY{n}{n\PYZus{}clusters\PYZus{}} \PY{o}{=} \PY{n+nb}{len}\PY{p}{(}\PY{n+nb}{set}\PY{p}{(}\PY{n}{labels}\PY{p}{)}\PY{p}{)} \PY{o}{\PYZhy{}} \PY{p}{(}\PY{l+m+mi}{1} \PY{k}{if} \PY{o}{\PYZhy{}}\PY{l+m+mi}{1} \PY{o+ow}{in} \PY{n}{labels} \PY{k}{else} \PY{l+m+mi}{0}\PY{p}{)}
\PY{n}{n\PYZus{}clusters\PYZus{}}
\end{Verbatim}
\end{tcolorbox}

            \begin{tcolorbox}[breakable, size=fbox, boxrule=.5pt, pad at break*=1mm, opacityfill=0]
\prompt{Out}{outcolor}{7}{\boxspacing}
\begin{Verbatim}[commandchars=\\\{\}]
3
\end{Verbatim}
\end{tcolorbox}
        
    \begin{tcolorbox}[breakable, size=fbox, boxrule=1pt, pad at break*=1mm,colback=cellbackground, colframe=cellborder]
\prompt{In}{incolor}{8}{\boxspacing}
\begin{Verbatim}[commandchars=\\\{\}]
\PY{c+c1}{\PYZsh{} Remove repetition in labels by turning it into a set.}
\PY{n}{unique\PYZus{}labels} \PY{o}{=} \PY{n+nb}{set}\PY{p}{(}\PY{n}{labels}\PY{p}{)}
\PY{n}{unique\PYZus{}labels}
\end{Verbatim}
\end{tcolorbox}

            \begin{tcolorbox}[breakable, size=fbox, boxrule=.5pt, pad at break*=1mm, opacityfill=0]
\prompt{Out}{outcolor}{8}{\boxspacing}
\begin{Verbatim}[commandchars=\\\{\}]
\{-1, 0, 1, 2\}
\end{Verbatim}
\end{tcolorbox}
        
    \hypertarget{data-visualization}{%
\subsubsection{Data visualization}\label{data-visualization}}

    \begin{tcolorbox}[breakable, size=fbox, boxrule=1pt, pad at break*=1mm,colback=cellbackground, colframe=cellborder]
\prompt{In}{incolor}{9}{\boxspacing}
\begin{Verbatim}[commandchars=\\\{\}]
\PY{c+c1}{\PYZsh{} Create colors for the clusters.}
\PY{n}{colors} \PY{o}{=} \PY{n}{plt}\PY{o}{.}\PY{n}{cm}\PY{o}{.}\PY{n}{Spectral}\PY{p}{(}\PY{n}{np}\PY{o}{.}\PY{n}{linspace}\PY{p}{(}\PY{l+m+mi}{0}\PY{p}{,} \PY{l+m+mi}{1}\PY{p}{,} \PY{n+nb}{len}\PY{p}{(}\PY{n}{unique\PYZus{}labels}\PY{p}{)}\PY{p}{)}\PY{p}{)}
\end{Verbatim}
\end{tcolorbox}

    \begin{tcolorbox}[breakable, size=fbox, boxrule=1pt, pad at break*=1mm,colback=cellbackground, colframe=cellborder]
\prompt{In}{incolor}{10}{\boxspacing}
\begin{Verbatim}[commandchars=\\\{\}]
\PY{c+c1}{\PYZsh{} Plot the points with colors}
\PY{k}{for} \PY{n}{k}\PY{p}{,} \PY{n}{col} \PY{o+ow}{in} \PY{n+nb}{zip}\PY{p}{(}\PY{n}{unique\PYZus{}labels}\PY{p}{,} \PY{n}{colors}\PY{p}{)}\PY{p}{:}
    \PY{k}{if} \PY{n}{k} \PY{o}{==} \PY{o}{\PYZhy{}}\PY{l+m+mi}{1}\PY{p}{:}
        \PY{c+c1}{\PYZsh{} Black used for noise.}
        \PY{n}{col} \PY{o}{=} \PY{l+s+s1}{\PYZsq{}}\PY{l+s+s1}{k}\PY{l+s+s1}{\PYZsq{}}

    \PY{n}{class\PYZus{}member\PYZus{}mask} \PY{o}{=} \PY{p}{(}\PY{n}{labels} \PY{o}{==} \PY{n}{k}\PY{p}{)}

    \PY{c+c1}{\PYZsh{} Plot the datapoints that are clustered}
    \PY{n}{xy} \PY{o}{=} \PY{n}{X}\PY{p}{[}\PY{n}{class\PYZus{}member\PYZus{}mask} \PY{o}{\PYZam{}} \PY{n}{core\PYZus{}samples\PYZus{}mask}\PY{p}{]}
    \PY{n}{plt}\PY{o}{.}\PY{n}{scatter}\PY{p}{(}\PY{n}{xy}\PY{p}{[}\PY{p}{:}\PY{p}{,} \PY{l+m+mi}{0}\PY{p}{]}\PY{p}{,} \PY{n}{xy}\PY{p}{[}\PY{p}{:}\PY{p}{,} \PY{l+m+mi}{1}\PY{p}{]}\PY{p}{,}\PY{n}{s}\PY{o}{=}\PY{l+m+mi}{50}\PY{p}{,} \PY{n}{c}\PY{o}{=}\PY{p}{[}\PY{n}{col}\PY{p}{]}\PY{p}{,} \PY{n}{marker}\PY{o}{=}\PY{l+s+sa}{u}\PY{l+s+s1}{\PYZsq{}}\PY{l+s+s1}{o}\PY{l+s+s1}{\PYZsq{}}\PY{p}{,} \PY{n}{alpha}\PY{o}{=}\PY{l+m+mf}{0.5}\PY{p}{)}

    \PY{c+c1}{\PYZsh{} Plot the outliers}
    \PY{n}{xy} \PY{o}{=} \PY{n}{X}\PY{p}{[}\PY{n}{class\PYZus{}member\PYZus{}mask} \PY{o}{\PYZam{}} \PY{o}{\PYZti{}}\PY{n}{core\PYZus{}samples\PYZus{}mask}\PY{p}{]}
    \PY{n}{plt}\PY{o}{.}\PY{n}{scatter}\PY{p}{(}\PY{n}{xy}\PY{p}{[}\PY{p}{:}\PY{p}{,} \PY{l+m+mi}{0}\PY{p}{]}\PY{p}{,} \PY{n}{xy}\PY{p}{[}\PY{p}{:}\PY{p}{,} \PY{l+m+mi}{1}\PY{p}{]}\PY{p}{,}\PY{n}{s}\PY{o}{=}\PY{l+m+mi}{50}\PY{p}{,} \PY{n}{c}\PY{o}{=}\PY{p}{[}\PY{n}{col}\PY{p}{]}\PY{p}{,} \PY{n}{marker}\PY{o}{=}\PY{l+s+sa}{u}\PY{l+s+s1}{\PYZsq{}}\PY{l+s+s1}{o}\PY{l+s+s1}{\PYZsq{}}\PY{p}{,} \PY{n}{alpha}\PY{o}{=}\PY{l+m+mf}{0.5}\PY{p}{)}
\end{Verbatim}
\end{tcolorbox}

    \begin{center}
    \adjustimage{max size={0.9\linewidth}{0.9\paperheight}}{output_17_0.png}
    \end{center}
    { \hspace*{\fill} \\}
    
    \hypertarget{practice}{%
\subsection{Practice}\label{practice}}

To better understand differences between partitional and density-based
clustering, try to cluster the above dataset into 3 clusters using
k-Means.\\
Notice: do not generate data again, use the same dataset as above.

    \begin{tcolorbox}[breakable, size=fbox, boxrule=1pt, pad at break*=1mm,colback=cellbackground, colframe=cellborder]
\prompt{In}{incolor}{11}{\boxspacing}
\begin{Verbatim}[commandchars=\\\{\}]
\PY{c+c1}{\PYZsh{} write your code here}
\PY{k+kn}{from} \PY{n+nn}{sklearn}\PY{n+nn}{.}\PY{n+nn}{cluster} \PY{k+kn}{import} \PY{n}{KMeans} 
\PY{n}{k} \PY{o}{=} \PY{l+m+mi}{3}
\PY{n}{k\PYZus{}means3} \PY{o}{=} \PY{n}{KMeans}\PY{p}{(}\PY{n}{init} \PY{o}{=} \PY{l+s+s2}{\PYZdq{}}\PY{l+s+s2}{k\PYZhy{}means++}\PY{l+s+s2}{\PYZdq{}}\PY{p}{,} \PY{n}{n\PYZus{}clusters} \PY{o}{=} \PY{n}{k}\PY{p}{,} \PY{n}{n\PYZus{}init} \PY{o}{=} \PY{l+m+mi}{12}\PY{p}{)}
\PY{n}{k\PYZus{}means3}\PY{o}{.}\PY{n}{fit}\PY{p}{(}\PY{n}{X}\PY{p}{)}
\PY{n}{fig} \PY{o}{=} \PY{n}{plt}\PY{o}{.}\PY{n}{figure}\PY{p}{(}\PY{n}{figsize}\PY{o}{=}\PY{p}{(}\PY{l+m+mi}{6}\PY{p}{,} \PY{l+m+mi}{4}\PY{p}{)}\PY{p}{)}
\PY{n}{ax} \PY{o}{=} \PY{n}{fig}\PY{o}{.}\PY{n}{add\PYZus{}subplot}\PY{p}{(}\PY{l+m+mi}{1}\PY{p}{,} \PY{l+m+mi}{1}\PY{p}{,} \PY{l+m+mi}{1}\PY{p}{)}
\PY{k}{for} \PY{n}{k}\PY{p}{,} \PY{n}{col} \PY{o+ow}{in} \PY{n+nb}{zip}\PY{p}{(}\PY{n+nb}{range}\PY{p}{(}\PY{n}{k}\PY{p}{)}\PY{p}{,} \PY{n}{colors}\PY{p}{)}\PY{p}{:}
    \PY{n}{my\PYZus{}members} \PY{o}{=} \PY{p}{(}\PY{n}{k\PYZus{}means3}\PY{o}{.}\PY{n}{labels\PYZus{}} \PY{o}{==} \PY{n}{k}\PY{p}{)}
    \PY{n}{plt}\PY{o}{.}\PY{n}{scatter}\PY{p}{(}\PY{n}{X}\PY{p}{[}\PY{n}{my\PYZus{}members}\PY{p}{,} \PY{l+m+mi}{0}\PY{p}{]}\PY{p}{,} \PY{n}{X}\PY{p}{[}\PY{n}{my\PYZus{}members}\PY{p}{,} \PY{l+m+mi}{1}\PY{p}{]}\PY{p}{,}  \PY{n}{color}\PY{o}{=}\PY{n}{col}\PY{p}{,} \PY{n}{marker}\PY{o}{=}\PY{l+s+sa}{u}\PY{l+s+s1}{\PYZsq{}}\PY{l+s+s1}{o}\PY{l+s+s1}{\PYZsq{}}\PY{p}{,} \PY{n}{alpha}\PY{o}{=}\PY{l+m+mf}{0.5}\PY{p}{)}
\PY{n}{plt}\PY{o}{.}\PY{n}{show}\PY{p}{(}\PY{p}{)}
\end{Verbatim}
\end{tcolorbox}

    \begin{Verbatim}[commandchars=\\\{\}]
C:\textbackslash{}Users\textbackslash{}stefo\textbackslash{}anaconda3\textbackslash{}lib\textbackslash{}site-packages\textbackslash{}sklearn\textbackslash{}cluster\textbackslash{}\_kmeans.py:1334:
UserWarning: KMeans is known to have a memory leak on Windows with MKL, when
there are less chunks than available threads. You can avoid it by setting the
environment variable OMP\_NUM\_THREADS=6.
  warnings.warn(
    \end{Verbatim}

    \begin{center}
    \adjustimage{max size={0.9\linewidth}{0.9\paperheight}}{output_19_1.png}
    \end{center}
    { \hspace*{\fill} \\}
    
    Click here for the solution

\begin{Shaded}
\begin{Highlighting}[]
\ImportTok{from}\NormalTok{ sklearn.cluster }\ImportTok{import}\NormalTok{ KMeans }
\NormalTok{k }\OperatorTok{=} \DecValTok{3}
\NormalTok{k\_means3 }\OperatorTok{=}\NormalTok{ KMeans(init }\OperatorTok{=} \StringTok{"k{-}means++"}\NormalTok{, n\_clusters }\OperatorTok{=}\NormalTok{ k, n\_init }\OperatorTok{=} \DecValTok{12}\NormalTok{)}
\NormalTok{k\_means3.fit(X)}
\NormalTok{fig }\OperatorTok{=}\NormalTok{ plt.figure(figsize}\OperatorTok{=}\NormalTok{(}\DecValTok{6}\NormalTok{, }\DecValTok{4}\NormalTok{))}
\NormalTok{ax }\OperatorTok{=}\NormalTok{ fig.add\_subplot(}\DecValTok{1}\NormalTok{, }\DecValTok{1}\NormalTok{, }\DecValTok{1}\NormalTok{)}
\ControlFlowTok{for}\NormalTok{ k, col }\KeywordTok{in} \BuiltInTok{zip}\NormalTok{(}\BuiltInTok{range}\NormalTok{(k), colors):}
\NormalTok{    my\_members }\OperatorTok{=}\NormalTok{ (k\_means3.labels\_ }\OperatorTok{==}\NormalTok{ k)}
\NormalTok{    plt.scatter(X[my\_members, }\DecValTok{0}\NormalTok{], X[my\_members, }\DecValTok{1}\NormalTok{],  c}\OperatorTok{=}\NormalTok{col, marker}\OperatorTok{=}\StringTok{u\textquotesingle{}o\textquotesingle{}}\NormalTok{, alpha}\OperatorTok{=}\FloatTok{0.5}\NormalTok{)}
\NormalTok{plt.show()}
\end{Highlighting}
\end{Shaded}

    Weather Station Clustering using DBSCAN \& scikit-learn

DBSCAN is especially very good for tasks like class identification in a
spatial context. The wonderful attribute of DBSCAN algorithm is that it
can find out any arbitrary shape cluster without getting affected by
noise. For example, this following example cluster the location of
weather stations in Canada. \textless Click 1\textgreater{} DBSCAN can
be used here, for instance, to find the group of stations which show the
same weather condition. As you can see, it not only finds different
arbitrary shaped clusters, can find the denser part of data-centered
samples by ignoring less-dense areas or noises.

Let's start playing with the data. We will be working according to the
following workflow:

\begin{enumerate}
\def\labelenumi{\arabic{enumi}.}
\tightlist
\item
  Loading data
\end{enumerate}

\begin{itemize}
\tightlist
\item
  Overview data
\item
  Data cleaning
\item
  Data selection
\item
  Clusteing
\end{itemize}

    \hypertarget{about-the-dataset}{%
\subsubsection{About the dataset}\label{about-the-dataset}}

Environment Canada\\
Monthly Values for July - 2015

Name in the table

Meaning

Stn\_Name

Station Name\textless/font

Lat

Latitude (North+, degrees)

Long

Longitude (West - , degrees)

Prov

Province

Tm

Mean Temperature (°C)

DwTm

Days without Valid Mean Temperature

D

Mean Temperature difference from Normal (1981-2010) (°C)

Tx

Highest Monthly Maximum Temperature (°C)

DwTx

Days without Valid Maximum Temperature

Tn

Lowest Monthly Minimum Temperature (°C)

DwTn

Days without Valid Minimum Temperature

S

Snowfall (cm)

DwS

Days without Valid Snowfall

S\%N

Percent of Normal (1981-2010) Snowfall

P

Total Precipitation (mm)

DwP

Days without Valid Precipitation

P\%N

Percent of Normal (1981-2010) Precipitation

S\_G

Snow on the ground at the end of the month (cm)

Pd

Number of days with Precipitation 1.0 mm or more

BS

Bright Sunshine (hours)

DwBS

Days without Valid Bright Sunshine

BS\%

Percent of Normal (1981-2010) Bright Sunshine

HDD

Degree Days below 18 °C

CDD

Degree Days above 18 °C

Stn\_No

Climate station identifier (first 3 digits indicate drainage basin, last
4 characters are for sorting alphabetically).

NA

Not Available

    \hypertarget{download-data}{%
\subsubsection{1-Download data}\label{download-data}}

To download the data, we will use \textbf{\texttt{!wget}} to download it
from IBM Object Storage.\\
\textbf{Did you know?} When it comes to Machine Learning, you will
likely be working with large datasets. As a business, where can you host
your data? IBM is offering a unique opportunity for businesses, with 10
Tb of IBM Cloud Object Storage:
\href{http://cocl.us/ML0101EN-IBM-Offer-CC}{Sign up now for free}

    \begin{tcolorbox}[breakable, size=fbox, boxrule=1pt, pad at break*=1mm,colback=cellbackground, colframe=cellborder]
\prompt{In}{incolor}{12}{\boxspacing}
\begin{Verbatim}[commandchars=\\\{\}]
\PY{o}{!}wget \PYZhy{}O weather\PYZhy{}stations20140101\PYZhy{}20141231.csv https://cf\PYZhy{}courses\PYZhy{}data.s3.us.cloud\PYZhy{}object\PYZhy{}storage.appdomain.cloud/IBMDeveloperSkillsNetwork\PYZhy{}ML0101EN\PYZhy{}SkillsNetwork/labs/Module\PYZpc{}204/data/weather\PYZhy{}stations20140101\PYZhy{}20141231.csv
\end{Verbatim}
\end{tcolorbox}

    \begin{Verbatim}[commandchars=\\\{\}]
--2022-10-12 13:05:14--  https://cf-courses-data.s3.us.cloud-object-
storage.appdomain.cloud/IBMDeveloperSkillsNetwork-ML0101EN-
SkillsNetwork/labs/Module\%204/data/weather-stations20140101-20141231.csv
Resolving cf-courses-data.s3.us.cloud-object-storage.appdomain.cloud (cf-
courses-data.s3.us.cloud-object-storage.appdomain.cloud){\ldots} 198.23.119.245
Connecting to cf-courses-data.s3.us.cloud-object-storage.appdomain.cloud (cf-
courses-data.s3.us.cloud-object-storage.appdomain.cloud)|198.23.119.245|:443{\ldots}
connected.
HTTP request sent, awaiting response{\ldots} 200 OK
Length: 129821 (127K) [text/csv]
Saving to: 'weather-stations20140101-20141231.csv'

     0K {\ldots} {\ldots} {\ldots} {\ldots} {\ldots} 39\%  315K 0s
    50K {\ldots} {\ldots} {\ldots} {\ldots} {\ldots} 78\%  721K 0s
   100K {\ldots} {\ldots} {\ldots}                          100\% 1.22M=0.2s

2022-10-12 13:05:15 (508 KB/s) - 'weather-stations20140101-20141231.csv' saved
[129821/129821]

    \end{Verbatim}

    \hypertarget{load-the-dataset}{%
\subsubsection{2- Load the dataset}\label{load-the-dataset}}

We will import the .csv then we creates the columns for year, month and
day.

    \begin{tcolorbox}[breakable, size=fbox, boxrule=1pt, pad at break*=1mm,colback=cellbackground, colframe=cellborder]
\prompt{In}{incolor}{13}{\boxspacing}
\begin{Verbatim}[commandchars=\\\{\}]
\PY{k+kn}{import} \PY{n+nn}{csv}
\PY{k+kn}{import} \PY{n+nn}{pandas} \PY{k}{as} \PY{n+nn}{pd}
\PY{k+kn}{import} \PY{n+nn}{numpy} \PY{k}{as} \PY{n+nn}{np}

\PY{n}{filename}\PY{o}{=}\PY{l+s+s1}{\PYZsq{}}\PY{l+s+s1}{weather\PYZhy{}stations20140101\PYZhy{}20141231.csv}\PY{l+s+s1}{\PYZsq{}}

\PY{c+c1}{\PYZsh{}Read csv}
\PY{n}{pdf} \PY{o}{=} \PY{n}{pd}\PY{o}{.}\PY{n}{read\PYZus{}csv}\PY{p}{(}\PY{n}{filename}\PY{p}{)}
\PY{n}{pdf}\PY{o}{.}\PY{n}{head}\PY{p}{(}\PY{l+m+mi}{5}\PY{p}{)}
\end{Verbatim}
\end{tcolorbox}

            \begin{tcolorbox}[breakable, size=fbox, boxrule=.5pt, pad at break*=1mm, opacityfill=0]
\prompt{Out}{outcolor}{13}{\boxspacing}
\begin{Verbatim}[commandchars=\\\{\}]
                 Stn\_Name     Lat     Long Prov   Tm  DwTm    D    Tx  DwTx  \textbackslash{}
0               CHEMAINUS  48.935 -123.742   BC  8.2   0.0  NaN  13.5   0.0
1  COWICHAN LAKE FORESTRY  48.824 -124.133   BC  7.0   0.0  3.0  15.0   0.0
2           LAKE COWICHAN  48.829 -124.052   BC  6.8  13.0  2.8  16.0   9.0
3        DISCOVERY ISLAND  48.425 -123.226   BC  NaN   NaN  NaN  12.5   0.0
4     DUNCAN KELVIN CREEK  48.735 -123.728   BC  7.7   2.0  3.4  14.5   2.0

    Tn  {\ldots}  DwP    P\%N  S\_G    Pd  BS  DwBS  BS\%    HDD  CDD   Stn\_No
0  1.0  {\ldots}  0.0    NaN  0.0  12.0 NaN   NaN  NaN  273.3  0.0  1011500
1 -3.0  {\ldots}  0.0  104.0  0.0  12.0 NaN   NaN  NaN  307.0  0.0  1012040
2 -2.5  {\ldots}  9.0    NaN  NaN  11.0 NaN   NaN  NaN  168.1  0.0  1012055
3  NaN  {\ldots}  NaN    NaN  NaN   NaN NaN   NaN  NaN    NaN  NaN  1012475
4 -1.0  {\ldots}  2.0    NaN  NaN  11.0 NaN   NaN  NaN  267.7  0.0  1012573

[5 rows x 25 columns]
\end{Verbatim}
\end{tcolorbox}
        
    \hypertarget{cleaning}{%
\subsubsection{3-Cleaning}\label{cleaning}}

Let's remove rows that don't have any value in the \textbf{Tm} field.

    \begin{tcolorbox}[breakable, size=fbox, boxrule=1pt, pad at break*=1mm,colback=cellbackground, colframe=cellborder]
\prompt{In}{incolor}{14}{\boxspacing}
\begin{Verbatim}[commandchars=\\\{\}]
\PY{n}{pdf} \PY{o}{=} \PY{n}{pdf}\PY{p}{[}\PY{n}{pd}\PY{o}{.}\PY{n}{notnull}\PY{p}{(}\PY{n}{pdf}\PY{p}{[}\PY{l+s+s2}{\PYZdq{}}\PY{l+s+s2}{Tm}\PY{l+s+s2}{\PYZdq{}}\PY{p}{]}\PY{p}{)}\PY{p}{]}
\PY{n}{pdf} \PY{o}{=} \PY{n}{pdf}\PY{o}{.}\PY{n}{reset\PYZus{}index}\PY{p}{(}\PY{n}{drop}\PY{o}{=}\PY{k+kc}{True}\PY{p}{)}
\PY{n}{pdf}\PY{o}{.}\PY{n}{head}\PY{p}{(}\PY{l+m+mi}{5}\PY{p}{)}
\end{Verbatim}
\end{tcolorbox}

            \begin{tcolorbox}[breakable, size=fbox, boxrule=.5pt, pad at break*=1mm, opacityfill=0]
\prompt{Out}{outcolor}{14}{\boxspacing}
\begin{Verbatim}[commandchars=\\\{\}]
                 Stn\_Name     Lat     Long Prov   Tm  DwTm    D    Tx  DwTx  \textbackslash{}
0               CHEMAINUS  48.935 -123.742   BC  8.2   0.0  NaN  13.5   0.0
1  COWICHAN LAKE FORESTRY  48.824 -124.133   BC  7.0   0.0  3.0  15.0   0.0
2           LAKE COWICHAN  48.829 -124.052   BC  6.8  13.0  2.8  16.0   9.0
3     DUNCAN KELVIN CREEK  48.735 -123.728   BC  7.7   2.0  3.4  14.5   2.0
4       ESQUIMALT HARBOUR  48.432 -123.439   BC  8.8   0.0  NaN  13.1   0.0

    Tn  {\ldots}  DwP    P\%N  S\_G    Pd  BS  DwBS  BS\%    HDD  CDD   Stn\_No
0  1.0  {\ldots}  0.0    NaN  0.0  12.0 NaN   NaN  NaN  273.3  0.0  1011500
1 -3.0  {\ldots}  0.0  104.0  0.0  12.0 NaN   NaN  NaN  307.0  0.0  1012040
2 -2.5  {\ldots}  9.0    NaN  NaN  11.0 NaN   NaN  NaN  168.1  0.0  1012055
3 -1.0  {\ldots}  2.0    NaN  NaN  11.0 NaN   NaN  NaN  267.7  0.0  1012573
4  1.9  {\ldots}  8.0    NaN  NaN  12.0 NaN   NaN  NaN  258.6  0.0  1012710

[5 rows x 25 columns]
\end{Verbatim}
\end{tcolorbox}
        
    \hypertarget{visualization}{%
\subsubsection{4-Visualization}\label{visualization}}

Visualization of stations on map using basemap package. The matplotlib
basemap toolkit is a library for plotting 2D data on maps in Python.
Basemap does not do any plotting on it's own, but provides the
facilities to transform coordinates to a map projections.

Please notice that the size of each data points represents the average
of maximum temperature for each station in a year.

    \begin{tcolorbox}[breakable, size=fbox, boxrule=1pt, pad at break*=1mm,colback=cellbackground, colframe=cellborder]
\prompt{In}{incolor}{15}{\boxspacing}
\begin{Verbatim}[commandchars=\\\{\}]
\PY{o}{!}pip install basemap
\end{Verbatim}
\end{tcolorbox}

    \begin{Verbatim}[commandchars=\\\{\}]
Requirement already satisfied: basemap in c:\textbackslash{}users\textbackslash{}stefo\textbackslash{}anaconda3\textbackslash{}lib\textbackslash{}site-
packages (1.3.3)
Requirement already satisfied: numpy<1.23,>=1.21 in
c:\textbackslash{}users\textbackslash{}stefo\textbackslash{}anaconda3\textbackslash{}lib\textbackslash{}site-packages (from basemap) (1.22.3)
Requirement already satisfied: pyproj<3.4.0,>=1.9.3 in
c:\textbackslash{}users\textbackslash{}stefo\textbackslash{}anaconda3\textbackslash{}lib\textbackslash{}site-packages (from basemap) (3.3.1)
Requirement already satisfied: matplotlib<3.6,>=1.5 in
c:\textbackslash{}users\textbackslash{}stefo\textbackslash{}anaconda3\textbackslash{}lib\textbackslash{}site-packages (from basemap) (3.4.3)
Requirement already satisfied: basemap-data<1.4,>=1.3.2 in
c:\textbackslash{}users\textbackslash{}stefo\textbackslash{}anaconda3\textbackslash{}lib\textbackslash{}site-packages (from basemap) (1.3.2)
Requirement already satisfied: pyshp<2.2,>=1.2 in
c:\textbackslash{}users\textbackslash{}stefo\textbackslash{}anaconda3\textbackslash{}lib\textbackslash{}site-packages (from basemap) (2.1.3)
Requirement already satisfied: cycler>=0.10 in
c:\textbackslash{}users\textbackslash{}stefo\textbackslash{}anaconda3\textbackslash{}lib\textbackslash{}site-packages (from matplotlib<3.6,>=1.5->basemap)
(0.10.0)
Requirement already satisfied: pyparsing>=2.2.1 in
c:\textbackslash{}users\textbackslash{}stefo\textbackslash{}anaconda3\textbackslash{}lib\textbackslash{}site-packages (from matplotlib<3.6,>=1.5->basemap)
(3.0.4)
Requirement already satisfied: pillow>=6.2.0 in
c:\textbackslash{}users\textbackslash{}stefo\textbackslash{}anaconda3\textbackslash{}lib\textbackslash{}site-packages (from matplotlib<3.6,>=1.5->basemap)
(8.4.0)
Requirement already satisfied: kiwisolver>=1.0.1 in
c:\textbackslash{}users\textbackslash{}stefo\textbackslash{}anaconda3\textbackslash{}lib\textbackslash{}site-packages (from matplotlib<3.6,>=1.5->basemap)
(1.3.1)
Requirement already satisfied: python-dateutil>=2.7 in
c:\textbackslash{}users\textbackslash{}stefo\textbackslash{}anaconda3\textbackslash{}lib\textbackslash{}site-packages (from matplotlib<3.6,>=1.5->basemap)
(2.8.2)
Requirement already satisfied: six in c:\textbackslash{}users\textbackslash{}stefo\textbackslash{}anaconda3\textbackslash{}lib\textbackslash{}site-packages
(from cycler>=0.10->matplotlib<3.6,>=1.5->basemap) (1.16.0)
Requirement already satisfied: certifi in c:\textbackslash{}users\textbackslash{}stefo\textbackslash{}anaconda3\textbackslash{}lib\textbackslash{}site-
packages (from pyproj<3.4.0,>=1.9.3->basemap) (2022.6.15)
    \end{Verbatim}

    \begin{tcolorbox}[breakable, size=fbox, boxrule=1pt, pad at break*=1mm,colback=cellbackground, colframe=cellborder]
\prompt{In}{incolor}{16}{\boxspacing}
\begin{Verbatim}[commandchars=\\\{\}]
\PY{k+kn}{from} \PY{n+nn}{mpl\PYZus{}toolkits}\PY{n+nn}{.}\PY{n+nn}{basemap} \PY{k+kn}{import} \PY{n}{Basemap}
\PY{k+kn}{import} \PY{n+nn}{matplotlib}\PY{n+nn}{.}\PY{n+nn}{pyplot} \PY{k}{as} \PY{n+nn}{plt}
\PY{k+kn}{from} \PY{n+nn}{pylab} \PY{k+kn}{import} \PY{n}{rcParams}
\PY{o}{\PYZpc{}}\PY{k}{matplotlib} inline
\PY{n}{rcParams}\PY{p}{[}\PY{l+s+s1}{\PYZsq{}}\PY{l+s+s1}{figure.figsize}\PY{l+s+s1}{\PYZsq{}}\PY{p}{]} \PY{o}{=} \PY{p}{(}\PY{l+m+mi}{14}\PY{p}{,}\PY{l+m+mi}{10}\PY{p}{)}

\PY{n}{llon}\PY{o}{=}\PY{o}{\PYZhy{}}\PY{l+m+mi}{140}
\PY{n}{ulon}\PY{o}{=}\PY{o}{\PYZhy{}}\PY{l+m+mi}{50}
\PY{n}{llat}\PY{o}{=}\PY{l+m+mi}{40}
\PY{n}{ulat}\PY{o}{=}\PY{l+m+mi}{65}

\PY{n}{pdf} \PY{o}{=} \PY{n}{pdf}\PY{p}{[}\PY{p}{(}\PY{n}{pdf}\PY{p}{[}\PY{l+s+s1}{\PYZsq{}}\PY{l+s+s1}{Long}\PY{l+s+s1}{\PYZsq{}}\PY{p}{]} \PY{o}{\PYZgt{}} \PY{n}{llon}\PY{p}{)} \PY{o}{\PYZam{}} \PY{p}{(}\PY{n}{pdf}\PY{p}{[}\PY{l+s+s1}{\PYZsq{}}\PY{l+s+s1}{Long}\PY{l+s+s1}{\PYZsq{}}\PY{p}{]} \PY{o}{\PYZlt{}} \PY{n}{ulon}\PY{p}{)} \PY{o}{\PYZam{}} \PY{p}{(}\PY{n}{pdf}\PY{p}{[}\PY{l+s+s1}{\PYZsq{}}\PY{l+s+s1}{Lat}\PY{l+s+s1}{\PYZsq{}}\PY{p}{]} \PY{o}{\PYZgt{}} \PY{n}{llat}\PY{p}{)} \PY{o}{\PYZam{}}\PY{p}{(}\PY{n}{pdf}\PY{p}{[}\PY{l+s+s1}{\PYZsq{}}\PY{l+s+s1}{Lat}\PY{l+s+s1}{\PYZsq{}}\PY{p}{]} \PY{o}{\PYZlt{}} \PY{n}{ulat}\PY{p}{)}\PY{p}{]}

\PY{n}{my\PYZus{}map} \PY{o}{=} \PY{n}{Basemap}\PY{p}{(}\PY{n}{projection}\PY{o}{=}\PY{l+s+s1}{\PYZsq{}}\PY{l+s+s1}{merc}\PY{l+s+s1}{\PYZsq{}}\PY{p}{,}
            \PY{n}{resolution} \PY{o}{=} \PY{l+s+s1}{\PYZsq{}}\PY{l+s+s1}{l}\PY{l+s+s1}{\PYZsq{}}\PY{p}{,} \PY{n}{area\PYZus{}thresh} \PY{o}{=} \PY{l+m+mf}{1000.0}\PY{p}{,}
            \PY{n}{llcrnrlon}\PY{o}{=}\PY{n}{llon}\PY{p}{,} \PY{n}{llcrnrlat}\PY{o}{=}\PY{n}{llat}\PY{p}{,} \PY{c+c1}{\PYZsh{}min longitude (llcrnrlon) and latitude (llcrnrlat)}
            \PY{n}{urcrnrlon}\PY{o}{=}\PY{n}{ulon}\PY{p}{,} \PY{n}{urcrnrlat}\PY{o}{=}\PY{n}{ulat}\PY{p}{)} \PY{c+c1}{\PYZsh{}max longitude (urcrnrlon) and latitude (urcrnrlat)}

\PY{n}{my\PYZus{}map}\PY{o}{.}\PY{n}{drawcoastlines}\PY{p}{(}\PY{p}{)}
\PY{n}{my\PYZus{}map}\PY{o}{.}\PY{n}{drawcountries}\PY{p}{(}\PY{p}{)}
\PY{c+c1}{\PYZsh{} my\PYZus{}map.drawmapboundary()}
\PY{n}{my\PYZus{}map}\PY{o}{.}\PY{n}{fillcontinents}\PY{p}{(}\PY{n}{color} \PY{o}{=} \PY{l+s+s1}{\PYZsq{}}\PY{l+s+s1}{white}\PY{l+s+s1}{\PYZsq{}}\PY{p}{,} \PY{n}{alpha} \PY{o}{=} \PY{l+m+mf}{0.3}\PY{p}{)}
\PY{n}{my\PYZus{}map}\PY{o}{.}\PY{n}{shadedrelief}\PY{p}{(}\PY{p}{)}

\PY{c+c1}{\PYZsh{} To collect data based on stations        }

\PY{n}{xs}\PY{p}{,}\PY{n}{ys} \PY{o}{=} \PY{n}{my\PYZus{}map}\PY{p}{(}\PY{n}{np}\PY{o}{.}\PY{n}{asarray}\PY{p}{(}\PY{n}{pdf}\PY{o}{.}\PY{n}{Long}\PY{p}{)}\PY{p}{,} \PY{n}{np}\PY{o}{.}\PY{n}{asarray}\PY{p}{(}\PY{n}{pdf}\PY{o}{.}\PY{n}{Lat}\PY{p}{)}\PY{p}{)}
\PY{n}{pdf}\PY{p}{[}\PY{l+s+s1}{\PYZsq{}}\PY{l+s+s1}{xm}\PY{l+s+s1}{\PYZsq{}}\PY{p}{]}\PY{o}{=} \PY{n}{xs}\PY{o}{.}\PY{n}{tolist}\PY{p}{(}\PY{p}{)}
\PY{n}{pdf}\PY{p}{[}\PY{l+s+s1}{\PYZsq{}}\PY{l+s+s1}{ym}\PY{l+s+s1}{\PYZsq{}}\PY{p}{]} \PY{o}{=}\PY{n}{ys}\PY{o}{.}\PY{n}{tolist}\PY{p}{(}\PY{p}{)}

\PY{c+c1}{\PYZsh{}Visualization1}
\PY{k}{for} \PY{n}{index}\PY{p}{,}\PY{n}{row} \PY{o+ow}{in} \PY{n}{pdf}\PY{o}{.}\PY{n}{iterrows}\PY{p}{(}\PY{p}{)}\PY{p}{:}
\PY{c+c1}{\PYZsh{}   x,y = my\PYZus{}map(row.Long, row.Lat)}
   \PY{n}{my\PYZus{}map}\PY{o}{.}\PY{n}{plot}\PY{p}{(}\PY{n}{row}\PY{o}{.}\PY{n}{xm}\PY{p}{,} \PY{n}{row}\PY{o}{.}\PY{n}{ym}\PY{p}{,}\PY{n}{markerfacecolor} \PY{o}{=}\PY{p}{(}\PY{p}{[}\PY{l+m+mi}{1}\PY{p}{,}\PY{l+m+mi}{0}\PY{p}{,}\PY{l+m+mi}{0}\PY{p}{]}\PY{p}{)}\PY{p}{,}  \PY{n}{marker}\PY{o}{=}\PY{l+s+s1}{\PYZsq{}}\PY{l+s+s1}{o}\PY{l+s+s1}{\PYZsq{}}\PY{p}{,} \PY{n}{markersize}\PY{o}{=} \PY{l+m+mi}{5}\PY{p}{,} \PY{n}{alpha} \PY{o}{=} \PY{l+m+mf}{0.75}\PY{p}{)}
\PY{c+c1}{\PYZsh{}plt.text(x,y,stn)}
\PY{n}{plt}\PY{o}{.}\PY{n}{show}\PY{p}{(}\PY{p}{)}
\end{Verbatim}
\end{tcolorbox}

    \begin{center}
    \adjustimage{max size={0.9\linewidth}{0.9\paperheight}}{output_31_0.png}
    \end{center}
    { \hspace*{\fill} \\}
    
    \hypertarget{clustering-of-stations-based-on-their-location-i.e.-lat-lon}{%
\subsubsection{5- Clustering of stations based on their location
i.e.~Lat \&
Lon}\label{clustering-of-stations-based-on-their-location-i.e.-lat-lon}}

\textbf{DBSCAN} form sklearn library can run DBSCAN clustering from
vector array or distance matrix. In our case, we pass it the Numpy array
Clus\_dataSet to find core samples of high density and expands clusters
from them.

    \begin{tcolorbox}[breakable, size=fbox, boxrule=1pt, pad at break*=1mm,colback=cellbackground, colframe=cellborder]
\prompt{In}{incolor}{17}{\boxspacing}
\begin{Verbatim}[commandchars=\\\{\}]
\PY{k+kn}{from} \PY{n+nn}{sklearn}\PY{n+nn}{.}\PY{n+nn}{cluster} \PY{k+kn}{import} \PY{n}{DBSCAN}
\PY{k+kn}{import} \PY{n+nn}{sklearn}\PY{n+nn}{.}\PY{n+nn}{utils}
\PY{k+kn}{from} \PY{n+nn}{sklearn}\PY{n+nn}{.}\PY{n+nn}{preprocessing} \PY{k+kn}{import} \PY{n}{StandardScaler}
\PY{n}{sklearn}\PY{o}{.}\PY{n}{utils}\PY{o}{.}\PY{n}{check\PYZus{}random\PYZus{}state}\PY{p}{(}\PY{l+m+mi}{1000}\PY{p}{)}
\PY{n}{Clus\PYZus{}dataSet} \PY{o}{=} \PY{n}{pdf}\PY{p}{[}\PY{p}{[}\PY{l+s+s1}{\PYZsq{}}\PY{l+s+s1}{xm}\PY{l+s+s1}{\PYZsq{}}\PY{p}{,}\PY{l+s+s1}{\PYZsq{}}\PY{l+s+s1}{ym}\PY{l+s+s1}{\PYZsq{}}\PY{p}{]}\PY{p}{]}
\PY{n}{Clus\PYZus{}dataSet} \PY{o}{=} \PY{n}{np}\PY{o}{.}\PY{n}{nan\PYZus{}to\PYZus{}num}\PY{p}{(}\PY{n}{Clus\PYZus{}dataSet}\PY{p}{)}
\PY{n}{Clus\PYZus{}dataSet} \PY{o}{=} \PY{n}{StandardScaler}\PY{p}{(}\PY{p}{)}\PY{o}{.}\PY{n}{fit\PYZus{}transform}\PY{p}{(}\PY{n}{Clus\PYZus{}dataSet}\PY{p}{)}

\PY{c+c1}{\PYZsh{} Compute DBSCAN}
\PY{n}{db} \PY{o}{=} \PY{n}{DBSCAN}\PY{p}{(}\PY{n}{eps}\PY{o}{=}\PY{l+m+mf}{0.15}\PY{p}{,} \PY{n}{min\PYZus{}samples}\PY{o}{=}\PY{l+m+mi}{10}\PY{p}{)}\PY{o}{.}\PY{n}{fit}\PY{p}{(}\PY{n}{Clus\PYZus{}dataSet}\PY{p}{)}
\PY{n}{core\PYZus{}samples\PYZus{}mask} \PY{o}{=} \PY{n}{np}\PY{o}{.}\PY{n}{zeros\PYZus{}like}\PY{p}{(}\PY{n}{db}\PY{o}{.}\PY{n}{labels\PYZus{}}\PY{p}{,} \PY{n}{dtype}\PY{o}{=}\PY{n+nb}{bool}\PY{p}{)}
\PY{n}{core\PYZus{}samples\PYZus{}mask}\PY{p}{[}\PY{n}{db}\PY{o}{.}\PY{n}{core\PYZus{}sample\PYZus{}indices\PYZus{}}\PY{p}{]} \PY{o}{=} \PY{k+kc}{True}
\PY{n}{labels} \PY{o}{=} \PY{n}{db}\PY{o}{.}\PY{n}{labels\PYZus{}}
\PY{n}{pdf}\PY{p}{[}\PY{l+s+s2}{\PYZdq{}}\PY{l+s+s2}{Clus\PYZus{}Db}\PY{l+s+s2}{\PYZdq{}}\PY{p}{]}\PY{o}{=}\PY{n}{labels}

\PY{n}{realClusterNum}\PY{o}{=}\PY{n+nb}{len}\PY{p}{(}\PY{n+nb}{set}\PY{p}{(}\PY{n}{labels}\PY{p}{)}\PY{p}{)} \PY{o}{\PYZhy{}} \PY{p}{(}\PY{l+m+mi}{1} \PY{k}{if} \PY{o}{\PYZhy{}}\PY{l+m+mi}{1} \PY{o+ow}{in} \PY{n}{labels} \PY{k}{else} \PY{l+m+mi}{0}\PY{p}{)}
\PY{n}{clusterNum} \PY{o}{=} \PY{n+nb}{len}\PY{p}{(}\PY{n+nb}{set}\PY{p}{(}\PY{n}{labels}\PY{p}{)}\PY{p}{)} 


\PY{c+c1}{\PYZsh{} A sample of clusters}
\PY{n}{pdf}\PY{p}{[}\PY{p}{[}\PY{l+s+s2}{\PYZdq{}}\PY{l+s+s2}{Stn\PYZus{}Name}\PY{l+s+s2}{\PYZdq{}}\PY{p}{,}\PY{l+s+s2}{\PYZdq{}}\PY{l+s+s2}{Tx}\PY{l+s+s2}{\PYZdq{}}\PY{p}{,}\PY{l+s+s2}{\PYZdq{}}\PY{l+s+s2}{Tm}\PY{l+s+s2}{\PYZdq{}}\PY{p}{,}\PY{l+s+s2}{\PYZdq{}}\PY{l+s+s2}{Clus\PYZus{}Db}\PY{l+s+s2}{\PYZdq{}}\PY{p}{]}\PY{p}{]}\PY{o}{.}\PY{n}{head}\PY{p}{(}\PY{l+m+mi}{5}\PY{p}{)}
\end{Verbatim}
\end{tcolorbox}

            \begin{tcolorbox}[breakable, size=fbox, boxrule=.5pt, pad at break*=1mm, opacityfill=0]
\prompt{Out}{outcolor}{17}{\boxspacing}
\begin{Verbatim}[commandchars=\\\{\}]
                 Stn\_Name    Tx   Tm  Clus\_Db
0               CHEMAINUS  13.5  8.2        0
1  COWICHAN LAKE FORESTRY  15.0  7.0        0
2           LAKE COWICHAN  16.0  6.8        0
3     DUNCAN KELVIN CREEK  14.5  7.7        0
4       ESQUIMALT HARBOUR  13.1  8.8        0
\end{Verbatim}
\end{tcolorbox}
        
    As you can see for outliers, the cluster label is -1

    \begin{tcolorbox}[breakable, size=fbox, boxrule=1pt, pad at break*=1mm,colback=cellbackground, colframe=cellborder]
\prompt{In}{incolor}{18}{\boxspacing}
\begin{Verbatim}[commandchars=\\\{\}]
\PY{n+nb}{set}\PY{p}{(}\PY{n}{labels}\PY{p}{)}
\end{Verbatim}
\end{tcolorbox}

            \begin{tcolorbox}[breakable, size=fbox, boxrule=.5pt, pad at break*=1mm, opacityfill=0]
\prompt{Out}{outcolor}{18}{\boxspacing}
\begin{Verbatim}[commandchars=\\\{\}]
\{-1, 0, 1, 2, 3, 4\}
\end{Verbatim}
\end{tcolorbox}
        
    \hypertarget{visualization-of-clusters-based-on-location}{%
\subsubsection{6- Visualization of clusters based on
location}\label{visualization-of-clusters-based-on-location}}

Now, we can visualize the clusters using basemap:

    \begin{tcolorbox}[breakable, size=fbox, boxrule=1pt, pad at break*=1mm,colback=cellbackground, colframe=cellborder]
\prompt{In}{incolor}{19}{\boxspacing}
\begin{Verbatim}[commandchars=\\\{\}]
\PY{k+kn}{from} \PY{n+nn}{mpl\PYZus{}toolkits}\PY{n+nn}{.}\PY{n+nn}{basemap} \PY{k+kn}{import} \PY{n}{Basemap}
\PY{k+kn}{import} \PY{n+nn}{matplotlib}\PY{n+nn}{.}\PY{n+nn}{pyplot} \PY{k}{as} \PY{n+nn}{plt}
\PY{k+kn}{from} \PY{n+nn}{pylab} \PY{k+kn}{import} \PY{n}{rcParams}
\PY{o}{\PYZpc{}}\PY{k}{matplotlib} inline
\PY{n}{rcParams}\PY{p}{[}\PY{l+s+s1}{\PYZsq{}}\PY{l+s+s1}{figure.figsize}\PY{l+s+s1}{\PYZsq{}}\PY{p}{]} \PY{o}{=} \PY{p}{(}\PY{l+m+mi}{14}\PY{p}{,}\PY{l+m+mi}{10}\PY{p}{)}

\PY{n}{my\PYZus{}map} \PY{o}{=} \PY{n}{Basemap}\PY{p}{(}\PY{n}{projection}\PY{o}{=}\PY{l+s+s1}{\PYZsq{}}\PY{l+s+s1}{merc}\PY{l+s+s1}{\PYZsq{}}\PY{p}{,}
            \PY{n}{resolution} \PY{o}{=} \PY{l+s+s1}{\PYZsq{}}\PY{l+s+s1}{l}\PY{l+s+s1}{\PYZsq{}}\PY{p}{,} \PY{n}{area\PYZus{}thresh} \PY{o}{=} \PY{l+m+mf}{1000.0}\PY{p}{,}
            \PY{n}{llcrnrlon}\PY{o}{=}\PY{n}{llon}\PY{p}{,} \PY{n}{llcrnrlat}\PY{o}{=}\PY{n}{llat}\PY{p}{,} \PY{c+c1}{\PYZsh{}min longitude (llcrnrlon) and latitude (llcrnrlat)}
            \PY{n}{urcrnrlon}\PY{o}{=}\PY{n}{ulon}\PY{p}{,} \PY{n}{urcrnrlat}\PY{o}{=}\PY{n}{ulat}\PY{p}{)} \PY{c+c1}{\PYZsh{}max longitude (urcrnrlon) and latitude (urcrnrlat)}

\PY{n}{my\PYZus{}map}\PY{o}{.}\PY{n}{drawcoastlines}\PY{p}{(}\PY{p}{)}
\PY{n}{my\PYZus{}map}\PY{o}{.}\PY{n}{drawcountries}\PY{p}{(}\PY{p}{)}
\PY{c+c1}{\PYZsh{}my\PYZus{}map.drawmapboundary()}
\PY{n}{my\PYZus{}map}\PY{o}{.}\PY{n}{fillcontinents}\PY{p}{(}\PY{n}{color} \PY{o}{=} \PY{l+s+s1}{\PYZsq{}}\PY{l+s+s1}{white}\PY{l+s+s1}{\PYZsq{}}\PY{p}{,} \PY{n}{alpha} \PY{o}{=} \PY{l+m+mf}{0.3}\PY{p}{)}
\PY{n}{my\PYZus{}map}\PY{o}{.}\PY{n}{shadedrelief}\PY{p}{(}\PY{p}{)}

\PY{c+c1}{\PYZsh{} To create a color map}
\PY{n}{colors} \PY{o}{=} \PY{n}{plt}\PY{o}{.}\PY{n}{get\PYZus{}cmap}\PY{p}{(}\PY{l+s+s1}{\PYZsq{}}\PY{l+s+s1}{jet}\PY{l+s+s1}{\PYZsq{}}\PY{p}{)}\PY{p}{(}\PY{n}{np}\PY{o}{.}\PY{n}{linspace}\PY{p}{(}\PY{l+m+mf}{0.0}\PY{p}{,} \PY{l+m+mf}{1.0}\PY{p}{,} \PY{n}{clusterNum}\PY{p}{)}\PY{p}{)}



\PY{c+c1}{\PYZsh{}Visualization1}
\PY{k}{for} \PY{n}{clust\PYZus{}number} \PY{o+ow}{in} \PY{n+nb}{set}\PY{p}{(}\PY{n}{labels}\PY{p}{)}\PY{p}{:}
    \PY{n}{c}\PY{o}{=}\PY{p}{(}\PY{p}{(}\PY{p}{[}\PY{l+m+mf}{0.4}\PY{p}{,}\PY{l+m+mf}{0.4}\PY{p}{,}\PY{l+m+mf}{0.4}\PY{p}{]}\PY{p}{)} \PY{k}{if} \PY{n}{clust\PYZus{}number} \PY{o}{==} \PY{o}{\PYZhy{}}\PY{l+m+mi}{1} \PY{k}{else} \PY{n}{colors}\PY{p}{[}\PY{n}{np}\PY{o}{.}\PY{n}{int}\PY{p}{(}\PY{n}{clust\PYZus{}number}\PY{p}{)}\PY{p}{]}\PY{p}{)}
    \PY{n}{clust\PYZus{}set} \PY{o}{=} \PY{n}{pdf}\PY{p}{[}\PY{n}{pdf}\PY{o}{.}\PY{n}{Clus\PYZus{}Db} \PY{o}{==} \PY{n}{clust\PYZus{}number}\PY{p}{]}                    
    \PY{n}{my\PYZus{}map}\PY{o}{.}\PY{n}{scatter}\PY{p}{(}\PY{n}{clust\PYZus{}set}\PY{o}{.}\PY{n}{xm}\PY{p}{,} \PY{n}{clust\PYZus{}set}\PY{o}{.}\PY{n}{ym}\PY{p}{,} \PY{n}{color} \PY{o}{=}\PY{n}{c}\PY{p}{,}  \PY{n}{marker}\PY{o}{=}\PY{l+s+s1}{\PYZsq{}}\PY{l+s+s1}{o}\PY{l+s+s1}{\PYZsq{}}\PY{p}{,} \PY{n}{s}\PY{o}{=} \PY{l+m+mi}{20}\PY{p}{,} \PY{n}{alpha} \PY{o}{=} \PY{l+m+mf}{0.85}\PY{p}{)}
    \PY{k}{if} \PY{n}{clust\PYZus{}number} \PY{o}{!=} \PY{o}{\PYZhy{}}\PY{l+m+mi}{1}\PY{p}{:}
        \PY{n}{cenx}\PY{o}{=}\PY{n}{np}\PY{o}{.}\PY{n}{mean}\PY{p}{(}\PY{n}{clust\PYZus{}set}\PY{o}{.}\PY{n}{xm}\PY{p}{)} 
        \PY{n}{ceny}\PY{o}{=}\PY{n}{np}\PY{o}{.}\PY{n}{mean}\PY{p}{(}\PY{n}{clust\PYZus{}set}\PY{o}{.}\PY{n}{ym}\PY{p}{)} 
        \PY{n}{plt}\PY{o}{.}\PY{n}{text}\PY{p}{(}\PY{n}{cenx}\PY{p}{,}\PY{n}{ceny}\PY{p}{,}\PY{n+nb}{str}\PY{p}{(}\PY{n}{clust\PYZus{}number}\PY{p}{)}\PY{p}{,} \PY{n}{fontsize}\PY{o}{=}\PY{l+m+mi}{25}\PY{p}{,} \PY{n}{color}\PY{o}{=}\PY{l+s+s1}{\PYZsq{}}\PY{l+s+s1}{red}\PY{l+s+s1}{\PYZsq{}}\PY{p}{,}\PY{p}{)}
        \PY{n+nb}{print} \PY{p}{(}\PY{l+s+s2}{\PYZdq{}}\PY{l+s+s2}{Cluster }\PY{l+s+s2}{\PYZdq{}}\PY{o}{+}\PY{n+nb}{str}\PY{p}{(}\PY{n}{clust\PYZus{}number}\PY{p}{)}\PY{o}{+}\PY{l+s+s1}{\PYZsq{}}\PY{l+s+s1}{, Avg Temp: }\PY{l+s+s1}{\PYZsq{}}\PY{o}{+} \PY{n+nb}{str}\PY{p}{(}\PY{n}{np}\PY{o}{.}\PY{n}{mean}\PY{p}{(}\PY{n}{clust\PYZus{}set}\PY{o}{.}\PY{n}{Tm}\PY{p}{)}\PY{p}{)}\PY{p}{)}
\end{Verbatim}
\end{tcolorbox}

    \begin{Verbatim}[commandchars=\\\{\}]
Cluster 0, Avg Temp: -5.538747553816051
Cluster 1, Avg Temp: 1.9526315789473685
Cluster 2, Avg Temp: -9.195652173913045
Cluster 3, Avg Temp: -15.300833333333333
Cluster 4, Avg Temp: -7.769047619047619
    \end{Verbatim}

    \begin{center}
    \adjustimage{max size={0.9\linewidth}{0.9\paperheight}}{output_37_1.png}
    \end{center}
    { \hspace*{\fill} \\}
    
    \hypertarget{clustering-of-stations-based-on-their-location-mean-max-and-min-temperature}{%
\subsubsection{7- Clustering of stations based on their location, mean,
max, and min
Temperature}\label{clustering-of-stations-based-on-their-location-mean-max-and-min-temperature}}

In this section we re-run DBSCAN, but this time on a 5-dimensional
dataset:

    \begin{tcolorbox}[breakable, size=fbox, boxrule=1pt, pad at break*=1mm,colback=cellbackground, colframe=cellborder]
\prompt{In}{incolor}{20}{\boxspacing}
\begin{Verbatim}[commandchars=\\\{\}]
\PY{k+kn}{from} \PY{n+nn}{sklearn}\PY{n+nn}{.}\PY{n+nn}{cluster} \PY{k+kn}{import} \PY{n}{DBSCAN}
\PY{k+kn}{import} \PY{n+nn}{sklearn}\PY{n+nn}{.}\PY{n+nn}{utils}
\PY{k+kn}{from} \PY{n+nn}{sklearn}\PY{n+nn}{.}\PY{n+nn}{preprocessing} \PY{k+kn}{import} \PY{n}{StandardScaler}
\PY{n}{sklearn}\PY{o}{.}\PY{n}{utils}\PY{o}{.}\PY{n}{check\PYZus{}random\PYZus{}state}\PY{p}{(}\PY{l+m+mi}{1000}\PY{p}{)}
\PY{n}{Clus\PYZus{}dataSet} \PY{o}{=} \PY{n}{pdf}\PY{p}{[}\PY{p}{[}\PY{l+s+s1}{\PYZsq{}}\PY{l+s+s1}{xm}\PY{l+s+s1}{\PYZsq{}}\PY{p}{,}\PY{l+s+s1}{\PYZsq{}}\PY{l+s+s1}{ym}\PY{l+s+s1}{\PYZsq{}}\PY{p}{,}\PY{l+s+s1}{\PYZsq{}}\PY{l+s+s1}{Tx}\PY{l+s+s1}{\PYZsq{}}\PY{p}{,}\PY{l+s+s1}{\PYZsq{}}\PY{l+s+s1}{Tm}\PY{l+s+s1}{\PYZsq{}}\PY{p}{,}\PY{l+s+s1}{\PYZsq{}}\PY{l+s+s1}{Tn}\PY{l+s+s1}{\PYZsq{}}\PY{p}{]}\PY{p}{]}
\PY{n}{Clus\PYZus{}dataSet} \PY{o}{=} \PY{n}{np}\PY{o}{.}\PY{n}{nan\PYZus{}to\PYZus{}num}\PY{p}{(}\PY{n}{Clus\PYZus{}dataSet}\PY{p}{)}
\PY{n}{Clus\PYZus{}dataSet} \PY{o}{=} \PY{n}{StandardScaler}\PY{p}{(}\PY{p}{)}\PY{o}{.}\PY{n}{fit\PYZus{}transform}\PY{p}{(}\PY{n}{Clus\PYZus{}dataSet}\PY{p}{)}

\PY{c+c1}{\PYZsh{} Compute DBSCAN}
\PY{n}{db} \PY{o}{=} \PY{n}{DBSCAN}\PY{p}{(}\PY{n}{eps}\PY{o}{=}\PY{l+m+mf}{0.3}\PY{p}{,} \PY{n}{min\PYZus{}samples}\PY{o}{=}\PY{l+m+mi}{10}\PY{p}{)}\PY{o}{.}\PY{n}{fit}\PY{p}{(}\PY{n}{Clus\PYZus{}dataSet}\PY{p}{)}
\PY{n}{core\PYZus{}samples\PYZus{}mask} \PY{o}{=} \PY{n}{np}\PY{o}{.}\PY{n}{zeros\PYZus{}like}\PY{p}{(}\PY{n}{db}\PY{o}{.}\PY{n}{labels\PYZus{}}\PY{p}{,} \PY{n}{dtype}\PY{o}{=}\PY{n+nb}{bool}\PY{p}{)}
\PY{n}{core\PYZus{}samples\PYZus{}mask}\PY{p}{[}\PY{n}{db}\PY{o}{.}\PY{n}{core\PYZus{}sample\PYZus{}indices\PYZus{}}\PY{p}{]} \PY{o}{=} \PY{k+kc}{True}
\PY{n}{labels} \PY{o}{=} \PY{n}{db}\PY{o}{.}\PY{n}{labels\PYZus{}}
\PY{n}{pdf}\PY{p}{[}\PY{l+s+s2}{\PYZdq{}}\PY{l+s+s2}{Clus\PYZus{}Db}\PY{l+s+s2}{\PYZdq{}}\PY{p}{]}\PY{o}{=}\PY{n}{labels}

\PY{n}{realClusterNum}\PY{o}{=}\PY{n+nb}{len}\PY{p}{(}\PY{n+nb}{set}\PY{p}{(}\PY{n}{labels}\PY{p}{)}\PY{p}{)} \PY{o}{\PYZhy{}} \PY{p}{(}\PY{l+m+mi}{1} \PY{k}{if} \PY{o}{\PYZhy{}}\PY{l+m+mi}{1} \PY{o+ow}{in} \PY{n}{labels} \PY{k}{else} \PY{l+m+mi}{0}\PY{p}{)}
\PY{n}{clusterNum} \PY{o}{=} \PY{n+nb}{len}\PY{p}{(}\PY{n+nb}{set}\PY{p}{(}\PY{n}{labels}\PY{p}{)}\PY{p}{)} 


\PY{c+c1}{\PYZsh{} A sample of clusters}
\PY{n}{pdf}\PY{p}{[}\PY{p}{[}\PY{l+s+s2}{\PYZdq{}}\PY{l+s+s2}{Stn\PYZus{}Name}\PY{l+s+s2}{\PYZdq{}}\PY{p}{,}\PY{l+s+s2}{\PYZdq{}}\PY{l+s+s2}{Tx}\PY{l+s+s2}{\PYZdq{}}\PY{p}{,}\PY{l+s+s2}{\PYZdq{}}\PY{l+s+s2}{Tm}\PY{l+s+s2}{\PYZdq{}}\PY{p}{,}\PY{l+s+s2}{\PYZdq{}}\PY{l+s+s2}{Clus\PYZus{}Db}\PY{l+s+s2}{\PYZdq{}}\PY{p}{]}\PY{p}{]}\PY{o}{.}\PY{n}{head}\PY{p}{(}\PY{l+m+mi}{5}\PY{p}{)}
\end{Verbatim}
\end{tcolorbox}

            \begin{tcolorbox}[breakable, size=fbox, boxrule=.5pt, pad at break*=1mm, opacityfill=0]
\prompt{Out}{outcolor}{20}{\boxspacing}
\begin{Verbatim}[commandchars=\\\{\}]
                 Stn\_Name    Tx   Tm  Clus\_Db
0               CHEMAINUS  13.5  8.2        0
1  COWICHAN LAKE FORESTRY  15.0  7.0        0
2           LAKE COWICHAN  16.0  6.8        0
3     DUNCAN KELVIN CREEK  14.5  7.7        0
4       ESQUIMALT HARBOUR  13.1  8.8        0
\end{Verbatim}
\end{tcolorbox}
        
    \hypertarget{visualization-of-clusters-based-on-location-and-temperture}{%
\subsubsection{8- Visualization of clusters based on location and
Temperture}\label{visualization-of-clusters-based-on-location-and-temperture}}

    \begin{tcolorbox}[breakable, size=fbox, boxrule=1pt, pad at break*=1mm,colback=cellbackground, colframe=cellborder]
\prompt{In}{incolor}{21}{\boxspacing}
\begin{Verbatim}[commandchars=\\\{\}]
\PY{k+kn}{from} \PY{n+nn}{mpl\PYZus{}toolkits}\PY{n+nn}{.}\PY{n+nn}{basemap} \PY{k+kn}{import} \PY{n}{Basemap}
\PY{k+kn}{import} \PY{n+nn}{matplotlib}\PY{n+nn}{.}\PY{n+nn}{pyplot} \PY{k}{as} \PY{n+nn}{plt}
\PY{k+kn}{from} \PY{n+nn}{pylab} \PY{k+kn}{import} \PY{n}{rcParams}
\PY{o}{\PYZpc{}}\PY{k}{matplotlib} inline
\PY{n}{rcParams}\PY{p}{[}\PY{l+s+s1}{\PYZsq{}}\PY{l+s+s1}{figure.figsize}\PY{l+s+s1}{\PYZsq{}}\PY{p}{]} \PY{o}{=} \PY{p}{(}\PY{l+m+mi}{14}\PY{p}{,}\PY{l+m+mi}{10}\PY{p}{)}

\PY{n}{my\PYZus{}map} \PY{o}{=} \PY{n}{Basemap}\PY{p}{(}\PY{n}{projection}\PY{o}{=}\PY{l+s+s1}{\PYZsq{}}\PY{l+s+s1}{merc}\PY{l+s+s1}{\PYZsq{}}\PY{p}{,}
            \PY{n}{resolution} \PY{o}{=} \PY{l+s+s1}{\PYZsq{}}\PY{l+s+s1}{l}\PY{l+s+s1}{\PYZsq{}}\PY{p}{,} \PY{n}{area\PYZus{}thresh} \PY{o}{=} \PY{l+m+mf}{1000.0}\PY{p}{,}
            \PY{n}{llcrnrlon}\PY{o}{=}\PY{n}{llon}\PY{p}{,} \PY{n}{llcrnrlat}\PY{o}{=}\PY{n}{llat}\PY{p}{,} \PY{c+c1}{\PYZsh{}min longitude (llcrnrlon) and latitude (llcrnrlat)}
            \PY{n}{urcrnrlon}\PY{o}{=}\PY{n}{ulon}\PY{p}{,} \PY{n}{urcrnrlat}\PY{o}{=}\PY{n}{ulat}\PY{p}{)} \PY{c+c1}{\PYZsh{}max longitude (urcrnrlon) and latitude (urcrnrlat)}

\PY{n}{my\PYZus{}map}\PY{o}{.}\PY{n}{drawcoastlines}\PY{p}{(}\PY{p}{)}
\PY{n}{my\PYZus{}map}\PY{o}{.}\PY{n}{drawcountries}\PY{p}{(}\PY{p}{)}
\PY{c+c1}{\PYZsh{}my\PYZus{}map.drawmapboundary()}
\PY{n}{my\PYZus{}map}\PY{o}{.}\PY{n}{fillcontinents}\PY{p}{(}\PY{n}{color} \PY{o}{=} \PY{l+s+s1}{\PYZsq{}}\PY{l+s+s1}{white}\PY{l+s+s1}{\PYZsq{}}\PY{p}{,} \PY{n}{alpha} \PY{o}{=} \PY{l+m+mf}{0.3}\PY{p}{)}
\PY{n}{my\PYZus{}map}\PY{o}{.}\PY{n}{shadedrelief}\PY{p}{(}\PY{p}{)}

\PY{c+c1}{\PYZsh{} To create a color map}
\PY{n}{colors} \PY{o}{=} \PY{n}{plt}\PY{o}{.}\PY{n}{get\PYZus{}cmap}\PY{p}{(}\PY{l+s+s1}{\PYZsq{}}\PY{l+s+s1}{jet}\PY{l+s+s1}{\PYZsq{}}\PY{p}{)}\PY{p}{(}\PY{n}{np}\PY{o}{.}\PY{n}{linspace}\PY{p}{(}\PY{l+m+mf}{0.0}\PY{p}{,} \PY{l+m+mf}{1.0}\PY{p}{,} \PY{n}{clusterNum}\PY{p}{)}\PY{p}{)}



\PY{c+c1}{\PYZsh{}Visualization1}
\PY{k}{for} \PY{n}{clust\PYZus{}number} \PY{o+ow}{in} \PY{n+nb}{set}\PY{p}{(}\PY{n}{labels}\PY{p}{)}\PY{p}{:}
    \PY{n}{c}\PY{o}{=}\PY{p}{(}\PY{p}{(}\PY{p}{[}\PY{l+m+mf}{0.4}\PY{p}{,}\PY{l+m+mf}{0.4}\PY{p}{,}\PY{l+m+mf}{0.4}\PY{p}{]}\PY{p}{)} \PY{k}{if} \PY{n}{clust\PYZus{}number} \PY{o}{==} \PY{o}{\PYZhy{}}\PY{l+m+mi}{1} \PY{k}{else} \PY{n}{colors}\PY{p}{[}\PY{n}{np}\PY{o}{.}\PY{n}{int}\PY{p}{(}\PY{n}{clust\PYZus{}number}\PY{p}{)}\PY{p}{]}\PY{p}{)}
    \PY{n}{clust\PYZus{}set} \PY{o}{=} \PY{n}{pdf}\PY{p}{[}\PY{n}{pdf}\PY{o}{.}\PY{n}{Clus\PYZus{}Db} \PY{o}{==} \PY{n}{clust\PYZus{}number}\PY{p}{]}                    
    \PY{n}{my\PYZus{}map}\PY{o}{.}\PY{n}{scatter}\PY{p}{(}\PY{n}{clust\PYZus{}set}\PY{o}{.}\PY{n}{xm}\PY{p}{,} \PY{n}{clust\PYZus{}set}\PY{o}{.}\PY{n}{ym}\PY{p}{,} \PY{n}{color} \PY{o}{=}\PY{n}{c}\PY{p}{,}  \PY{n}{marker}\PY{o}{=}\PY{l+s+s1}{\PYZsq{}}\PY{l+s+s1}{o}\PY{l+s+s1}{\PYZsq{}}\PY{p}{,} \PY{n}{s}\PY{o}{=} \PY{l+m+mi}{20}\PY{p}{,} \PY{n}{alpha} \PY{o}{=} \PY{l+m+mf}{0.85}\PY{p}{)}
    \PY{k}{if} \PY{n}{clust\PYZus{}number} \PY{o}{!=} \PY{o}{\PYZhy{}}\PY{l+m+mi}{1}\PY{p}{:}
        \PY{n}{cenx}\PY{o}{=}\PY{n}{np}\PY{o}{.}\PY{n}{mean}\PY{p}{(}\PY{n}{clust\PYZus{}set}\PY{o}{.}\PY{n}{xm}\PY{p}{)} 
        \PY{n}{ceny}\PY{o}{=}\PY{n}{np}\PY{o}{.}\PY{n}{mean}\PY{p}{(}\PY{n}{clust\PYZus{}set}\PY{o}{.}\PY{n}{ym}\PY{p}{)} 
        \PY{n}{plt}\PY{o}{.}\PY{n}{text}\PY{p}{(}\PY{n}{cenx}\PY{p}{,}\PY{n}{ceny}\PY{p}{,}\PY{n+nb}{str}\PY{p}{(}\PY{n}{clust\PYZus{}number}\PY{p}{)}\PY{p}{,} \PY{n}{fontsize}\PY{o}{=}\PY{l+m+mi}{25}\PY{p}{,} \PY{n}{color}\PY{o}{=}\PY{l+s+s1}{\PYZsq{}}\PY{l+s+s1}{red}\PY{l+s+s1}{\PYZsq{}}\PY{p}{,}\PY{p}{)}
        \PY{n+nb}{print} \PY{p}{(}\PY{l+s+s2}{\PYZdq{}}\PY{l+s+s2}{Cluster }\PY{l+s+s2}{\PYZdq{}}\PY{o}{+}\PY{n+nb}{str}\PY{p}{(}\PY{n}{clust\PYZus{}number}\PY{p}{)}\PY{o}{+}\PY{l+s+s1}{\PYZsq{}}\PY{l+s+s1}{, Avg Temp: }\PY{l+s+s1}{\PYZsq{}}\PY{o}{+} \PY{n+nb}{str}\PY{p}{(}\PY{n}{np}\PY{o}{.}\PY{n}{mean}\PY{p}{(}\PY{n}{clust\PYZus{}set}\PY{o}{.}\PY{n}{Tm}\PY{p}{)}\PY{p}{)}\PY{p}{)}
\end{Verbatim}
\end{tcolorbox}

    \begin{Verbatim}[commandchars=\\\{\}]
Cluster 0, Avg Temp: 6.2211920529801334
Cluster 1, Avg Temp: 6.790000000000001
Cluster 2, Avg Temp: -0.49411764705882355
Cluster 3, Avg Temp: -13.877209302325586
Cluster 4, Avg Temp: -4.186274509803922
Cluster 5, Avg Temp: -16.301503759398482
Cluster 6, Avg Temp: -13.599999999999998
Cluster 7, Avg Temp: -9.753333333333334
Cluster 8, Avg Temp: -4.258333333333334
    \end{Verbatim}

    \begin{center}
    \adjustimage{max size={0.9\linewidth}{0.9\paperheight}}{output_41_1.png}
    \end{center}
    { \hspace*{\fill} \\}
    
    Want to learn more?

IBM SPSS Modeler is a comprehensive analytics platform that has many
machine learning algorithms. It has been designed to bring predictive
intelligence to decisions made by individuals, by groups, by systems --
by your enterprise as a whole. A free trial is available through this
course, available here: SPSS Modeler

Also, you can use Watson Studio to run these notebooks faster with
bigger datasets. Watson Studio is IBM's leading cloud solution for data
scientists, built by data scientists. With Jupyter notebooks, RStudio,
Apache Spark and popular libraries pre-packaged in the cloud, Watson
Studio enables data scientists to collaborate on their projects without
having to install anything. Join the fast-growing community of Watson
Studio users today with a free account at Watson Studio

    \hypertarget{thank-you-for-completing-this-lab}{%
\subsubsection{Thank you for completing this
lab!}\label{thank-you-for-completing-this-lab}}

\hypertarget{author}{%
\subsection{Author}\label{author}}

Saeed Aghabozorgi

\hypertarget{other-contributors}{%
\subsubsection{Other Contributors}\label{other-contributors}}

Joseph Santarcangelo

\hypertarget{change-log}{%
\subsection{Change Log}\label{change-log}}



\#\#

© IBM Corporation 2020. All rights reserved.

    \begin{tcolorbox}[breakable, size=fbox, boxrule=1pt, pad at break*=1mm,colback=cellbackground, colframe=cellborder]
\prompt{In}{incolor}{ }{\boxspacing}
\begin{Verbatim}[commandchars=\\\{\}]

\end{Verbatim}
\end{tcolorbox}


    % Add a bibliography block to the postdoc
    
    
    
\end{document}
